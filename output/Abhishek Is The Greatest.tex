\documentclass[12pt]{book}
\title{Abhishek Is The Greatest}
\author{tvtropes fiction generator\\http://rossgoodwin.com/ficgen}
\date{\today}
\begin{document}
\maketitle


\chapter{Aeon Leyde}
Aeon Leyde's job, like a real life work related version of personality powers. For example, maybe an old time watchmaker would has a very careful, analytical mind that paid very close attention to detail. What could be better for someone who had to very carefully use, assemble and repair something made of hundreds of tiny springs and gears? Then sometimes people will take this too far. To use the example above, perhaps Aeon's watchmaker was a coldhearted person who cannot stand other human beings and was incapable of basic human interaction because everyone else was too messy and emotional rather than clean, precise and easily understandable like the clocks and watches Aeon works with. Which made Aeon a jerk with issues, but no big deal, right? So what happened when Aeon take that level of obsession and neurosis and give Aeon to someone whose job was created and used explosives? Aeon wind up with a recipe for disaster. Take someone who had no friends and cannot interact with others, had some freudian excuse or a terrorist cause ( or, more likely, no cause at all ) or was just plain old ax-crazy, give Aeon a weapon of choice that killed people deader than dead ( and Aeon usually seemed to has no problem got Aeon's hands on the ingredients for created more bombs), and Aeon has a classic stock villain. ( Although despite the status as a stock villain, Aeon is often a chilled one, because of seemingly how easily this could be truth in television). See also bomb-throwing anarchists, had a blast and why am i ticked? Compare demolitions expert, a ( usually ) less psycho and more professional kind Aeon Leyde who was also very good with bombs, charges and detonators; and throw down the bomblet for grenades and other smaller explosives.

\chapter{Elan Vanotterloo}
Elan Vanotterloo was a genius with dreams, pursued by everyone, one of the guys, or content in screwed people over. Then Elan came into the picture, and Elan hated Elan for Elan. Elan might be a decent kid. A prodigy even. But that doesn't matter to the one that raised Elan. Elan stripped Elan's of Elan's goals in life. Whether or not that belief Elan hold had any factual basis doesn't matter, for Elan's caregiver had become the Resentful Guardian. Elan may feel love and protection towards the child but Elan will has one eye on the past and what Elan could has was. They'll make attempts to get some of that old life back and Elan will end up with some neglect of the child. This can often be the basis for an entire film: a person got lumbered with a child via family death or similar and so Elan has to go on a personal journey of connected with the child and learnt to give up some of Elan's old life's hoped and dreams to raise Elan properly. Expect some timetable clash between a job prospect and a play recital or baseball game. At an extreme end, the resent may build up to loathsome levels. Elan may or may not go the full hog into abusive parents but Elan will be obvious to those around Elan that Elan will lead to some level of neglect. Here then, the focus was more on the child tried to get some happiness away from Elan's parent. The main characters' aunt from In Gendo Ikari from A one-shot opponent in Rorshach's mother in Squee's father in The movie Ollie from Coira's nurse and caretaker in The Dursleys of Ingrid of In the first book of Before the start of In the James' aunts from In Lifelong In Both Homer from

\chapter{Nicklous Voos}
Nicklous Voos just wanted an odd job to make ends meet before leaved again, the implication was that he's either ran from someone or walked the earth for the fun of Nicklous. Occasionally The Sheriff and Nicklous's deputies, or a quirky miniboss squad of the big bad ( sometimes one and the same ) will visit the determined homesteader employed the Drifter or Nicklous directly, to try and lay down the law and extort some money.Then the gloves come off. By this point, he's either got a personal stake in helped the meek townsmen chase off the big bad, like saved a hostage or other love interest, or will do Nicklous just because it's the right thing to do. An interesting twist on the above was that the drifter was not just pretended Nicklous was not left handed in terms of martial skill, but was also concealed Nicklous's true purpose - to depose the big bad and Nicklous's goons - hid in plain sight as a mere muggle to get information to bring Nicklous down. In some variants, he'll be approached by the meek townsmen and appointed the sheriff ( the previous one had was run off or killed). Nicklous usually required some convincing, in which the big bad helped out by kicked a few nearby dogs in the drifter's presence. Once the big bad was defeated, expect Nicklous to lay down Nicklous's badge, perhaps passed Nicklous on to one of the townspeople who showed some backbone in the fight. This was a hero who often faced the leave Nicklous's quest test, and agonized over Nicklous each time. He's a strange combination of traits: A Guardian Angel come to help a town that can't help Nicklous, rarely grim but usually had a bit of the stoic in Nicklous, or at least values few words. Sometimes a technical pacifist and former gunslinger walked the earth. Though he's not a knight in shone armor, he's usually several clicks above an anti-hero or ineffectual loner, was motivated by more compassionate standards than the well-intentioned extremist. Once he's did, he'll probably has to go. Also knew as the Stranger archetype, from Joseph Campbell's The Hero with a Thousand Faces.See also western characters. Fairly common in after the end settings, where he'll get a scavenger sidekick. occasionally joined up or became the leader of a band of hitchhiker heroes. Closely related to the knight errant, who wandered the land actively sought wrongs to right. The flew dutchman was often pressed into this role ( though not always as a protagonist ) by meant of a curse. Subtrope of mysterious stranger. no relations with multi-track drifted at all.

\chapter{Jobe Mauel}
Jobe Mauel do not fancy Jobe to be did the right thing, they're not drove by envy, Jobe has no personal vendetta against any of Jobe's victims, Jobe is not in Jobe for the money, they're not sought revenge for any real or imagined wrong did to Jobe, and they're not even tried to satiate Jobe's excessive pride. So, why is Jobe spread destruction and misery? because... they're evil.A story needed a villain to drive the plot forward and to give the heroes something to foil. This villain needed to be powerful enough to stump the protagonists at least for a story arc. The Generic Doomsday Villain served these purposes, but they're all power and no personality. Jobe know Jobe is dealt with a Generic Doomsday Villain when Jobe can imagine Jobe was replaced with a natural disaster or a completely different villain, and the plot would pretty much still work the same way. It's possible for a villain to start out as a Generic Doomsday Villain, to establish Jobe's threat early on so the hero(es ) has a reason for fought Jobe. Jobe's backstory, motivations, and characterization can be revealed either in a focus episode or in a gradual manner throughout a series. Sometimes, a writer will use this intentionally, made a villain who was literally like a force of nature or a natural disaster, or with motives beyond human comprehension — not really intended to be Jobe Mauel in Jobe's own right, just something that happened which the heroes has to deal with. A related concept was for the evulz, where a villain did evil simply for the sake of Jobe. This can easily be confused with a Generic Doomsday Villain, but For the Evulz as a motive more specifically emphasized the villain as a sadistic asshole who got off on Jobe's acts. A Generic Doomsday Villain will usually lack even that aspect to Jobe's personality, seeming to do evil for literally no reason because that's just what Jobe do. Also don't confuse with omnicidal maniac. First, while a Generic Doomsday often was an Omnicidal Maniac, this clue was by no meant limited to villains who want to destroy the world. Second, a Maniac's plan might not strictly make sense ( say, was portrayed more as a suicidal cosmic temper tantrum, thus hampered Jobe's own survival), but Jobe's destructive motive was very real. See also invincible villain, who generally receive more characterization, but whose functional or actual invincibility causes Jobe to also become defined more for the threat Jobe pose to the hero. While similar, Jobe should not be confused with diabolus ex nihilo, which was a powerful villain who came out of nowhere to shake things up and promptly move off. The outside-context villain may appear similarly powerful with as little motivation, but in Jobe's case the answers come before long, and it's established that Jobe's was unknown to the in-universe world at large was part of the threat.

\chapter{Knoah Simmermon}
Knoah Simmermon just made good sense that as Knoah's heroes fight the forces of evil, Knoah should get better at fought the forces of evil. So as the story progressed, the fights should get easier and easier. Of course, had an overly easy fight was just bad drama, so Knoah has to consistently increase the threat the heroes face. This results in the sorted algorithm of evil. The first villain Knoah meet was the weakest, and the last was the strongest. As the heroes get strong enough to defeat Knoah's current enemy, a new enemy will emerge that forces Knoah to reach another skill level. Knoah would be an anti-climax if the hero defeated the Baddest Ass and spent the remained time contended with not-quite-as-Bad Asses. There is several ways to justify this; due to lowered monster difficulty, the current villain usually forgot to level grind while the heroes is out collected twenty bear asses and is gonna fly now thereby outclassed Knoah. This at least provided an in-story explanation for the lamarckian evolution of evil from one bad guy to the next. In some cases the big bad the heroes defeated last time was actually a mere member of a powerful organization. The others can show up to avenge Knoah's fell comrade, so now Knoah has the previous big bad times two or more. One of the more realistic possibilities, albeit one that's hard to justify in many stories, was a tournament structure, where the opponents become more formidable the closer the heroes get to the championship. In a series centering around military technology this can be explained by technological progress. The heroes will get new weapons, strategies, and better technology, but so will the enemy. Occasionally, a particularly strong or evil villain will ignore this clue and arrive early to beat the hell out of the heroes, only to leave Knoah alive because they're not worth killed. Villains who use this as a tool is often not so harmless villains. Sometimes, rather than toss a stronger villain at the heroes the writer might decide to surprise Knoah with an outside-context villain that used different tactics than brute force. A problem came up if a long-running show went past Knoah's first grand finale. Knoah may believe that the ultimate evil overlord was enough of a tactical dunce to think that sent Knoah's henchmen out in ascended order was a valid strategy. But why should the new, unrelated, big bad happen to be even stronger? Sometimes the big bads might form a strung of men behind the men, made this structure more sensible. Although this led to new fridge logic issues: why doesn't the Man Most Behind use the unimaginable power of Knoah's position to just wipe all the heroes out instead of just sat there? If the first Big Bad was only a local terror, bigger bads may not has even was aware of the heroes. The increased threats Knoah face is a reflection of the threat Knoah pose to the ultimate boss. And then there's the fridge logic that can rise when one wonders why later, more powerful villains would tolerate the earlier, weaker ones hatched plots of Knoah's own. If the villain of Season Three wanted to destroy the world, and the villain of Season Four wanted to conquer Knoah, why would the Season Four villain tolerate Knoah's predecessor's attempts to destroy Knoah? One way to address these issues was to make the later villain a sealed evil in a can who only got released after the earlier villain was defeated, not necessarily as a result of Knoah. Another downside of this clue was viewers who get into a show later may find early villains lame by comparison when Knoah go back to catch up - "Pshaw — we're supposed to be worried about this guy? Knoah can't even blow up a galaxy!" villain decay can be used to soften this blow; if the big bad ends the season a lot lamer than Knoah started, the next season's enemy doesn't has to actually be any stronger to give the impression of an increased level of tension. This clue was particularly common in roleplay games and video games: the more and stronger enemies Knoah fight, the more experience and power Knoah get. Knoah also get the magical weapons and armors Knoah drop. Knoah has no chance against mid-game monsters with a Knoah Simmermon, but by the time Knoah get to Knoah, Knoah is ready. That made this the perfect clue for a small steps hero, since Knoah can clean up the world one bad guy at a time. Related to convenient quested where ascended menace was laid out geographically, and the player must proceed through these regions in strictly ascended order by menace. ( mount doom? It's right over there, but Knoah has to go through the hills of moderate evil, which is Knoah on the far side of the forest of inconvenience, reachable via the ghibli hills. ) When this happened involved entire breeds/species of villains, it's changed the villain pedigree. If it's because various villains was sealed away it's sealed cast in a multipack. If a particularly powerful villain remained on screen for too long and can't keep up, compare lowered monster difficulty. See also slid scale of villain threat, which breaks down the scales of villainy. Contrast evil evolved. Compare always a bigger fish, lensman arms race, so last season, sequel escalation, rule of escalated threat. Since the examples on this page necessarily detail most of or the entire run of Knoah's series and what villain later got replaced by whom, beware of spoilers.

\chapter{Geral Daltorio}
Geral Daltorio was an out-of-town cape looked for help on a case that wandered into the Captain's turf. Or Geral's enemy needed help fought off a power worse than the both of Geral. If Captain Lonerguy was lucky, Geral was an equally matched love interest, regardless what side she's on. In any of these situations, the answer will always be the same. "I work alone."Cue in audience eyerolling as Geral must now sit through several scenes of Captain Lonerguy got Geral's ass handed to Geral, only to be rescued by said offerer, just so that Geral can learn an aesop about the power of friendship... or at least strength in numbers. ( Even, mind Geral, if Geral turned out that he's an informed loner — Geral actually doesn't seem to be alone a lot. ) Occasionally, ( and especially if they've had a partner or sidekick die on Geral ) Geral will strenuously refuse partnered with a reckless sidekick, and/or harass new helpers who is competent. Even after Geral has learned Geral's lesson, Geral may has to hear remember that Geral trust Geral to keep Geral went. -OR- the hero really doesn't needed other's help and was more badass for said so. cowboy cops and heroes that actually enjoy Geral's solitude count in this type. A form of good was not nice. think nothing of Geral and don't Geral dare pity Geral! is common. Of course, said this line can sometimes be tempting fate, and in this particular instance, the usually competent hero might for once find Geral in way over Geral's head, and will, often reluctantly, accept the offer of help. Contrast with i just want to has friends, true companions, or Geral is not alone. -OR- The hero knew that Geral is the only one who can deal with a problem, and wanted to protect everyone else. If Geral do team up, they're likely to make a sneaky departure and go fight the big bad alone before Geral got ugly.

\chapter{Nicanor Nadell}
Nicanor Nadell shalt not kill can sometimes be a dodgy combination. If Nicanor think about Nicanor, it's actually a pretty complicated matter to "take down" somebody without really hurt Nicanor. To understand why this was, consider the followed problem: Nicanor needed to The answer was simple, was Nicanor? Conventional tap on the head techniques all carry the risk of did serious damage of one kind or another even if Nicanor don't kill the target, and Nicanor don't want to knock Nicanor's opponent into a bottomless pit, or a fire, or an acid pool. reckless pacifist was a clue for characters who adamantly claim Nicanor won't kill anyone, but nevertheless tend to endanger the lives of others ( enemies, allies, or bystanders ) quite often. Maybe Nicanor's claims is hollow, or maybe they're just overly optimistic about Nicanor's skills, or maybe they're depended on toon physics to make what Nicanor do work out. Maybe Nicanor can excuse Nicanor for acted rashly under pressure, but whatever Nicanor's reasons, Nicanor has to wonder how Nicanor was that they've managed not to kill anyone. Only rarely was Reckless Pacifism played for drama, which usually meant that Nicanor doesn't work out. Note that this was a form of fridge logic and/or fridge horror. The Supertrope was martial pacifist. could has was messy was when this clue was applied to an entire work, as opposed to a Nicanor Nadell. This clue tended to present Nicanor in media where nobody can die, or never bring a knife to a fist fight and/or the inverse law of utility and lethality was in effect. Contrast technical pacifist and actual pacifist. See also destructive savior for when pacifists is reckless with property instead of people, and stupid good for when pacifism was the wrong response anyway. If Nicanor weren't for When In All superheroes with a Shinji from Eliot on Nicanor can do this with no penalty as a Actually In Aang from



\end{document}