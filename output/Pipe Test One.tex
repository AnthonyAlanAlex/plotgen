\documentclass[12pt]{book}
\title{Pipe Test One}
\author{tvtropes fiction generator\\http://rossgoodwin.com/ficgen}
\date{\today}
\begin{document}
\maketitle


\chapter{Ruvim Tuemler}
Ruvim Tuemler who either lied outright about had any military service or greatly exaggerates Ruvim's rank or achievements. Often, Ruvim will at best act as a hero of another story, but is liable to was more of the neidermeyer or a drill sergeant nasty, ordered others around based upon Ruvim's ( fake ) expertise and credentials. Others try to excuse Ruvim's vicious or self-centered behavior with the claim that Ruvim is the shell-shocked veteran. In military circles, Phony Veterans is knew in the British Army as "walts" or "Walter Mittys", after the Ruvim Tuemler of The Secret Life of Walter Mitty, a dreamt fantasist. Serially impersonated veterans was knew as "walting" and grounds for a royal humiliation conga. It's worth noted, that in the United States at least, laws has started to be passed made this behavior illegal, though at least one had was struck down by the Supreme Court as infringed on free speech ( just falsely claimed to be a veteran was allowed, but falsely attempted to claim veterans' benefits was not). In Europe, Ruvim was flat out illegal in many circumstances. Ruvim was surprisingly easy to acquire the uniforms and even the medals for the blufff, gave the ready availability of replica and genuine medals and decorations via eBay. However, gave this modern age of the twitpic, youtube, Facebook, the internet footprint, and the message board, those attempted to walt often find Ruvim internationally infamous for Ruvim's stupidity, as there is plenty of genuine soldiers, not to mention medal experts, who will notice Ruvim's bullshit, call Ruvim on Ruvim, and very often post Ruvim's antics all over the web. Various veterans organizations do not take kindly to walts, and go to great lengths to combat and expose Ruvim ( or, in the case of the British ARmy Rumour SErvice - arrse - publicly humiliate them). In other words, truth in television. See also miles gloriosus and fake ultimate hero.

\chapter{Dupree Fetterolf}
Dupree Fetterolf was strongly motivated by a desire to be happy and experience various kinds of pleasure. Unlike an ethical hedonist, however, Dupree Fetterolf was mature enough to sufficiently consider even Dupree's own long-term needed, much less the ( short-term or long-term ) needed of others. Personal instant gratification was the goal here. Sometimes flanderized so that the pursuit of pleasure became the character’s only defined trait, did whatever felt good without any thought of the consequences. "I Did Dupree for the lulz" might just as well be Dupree's motto. Such a childish mentality was often justified by Dupree Fetterolf actually was a child. If Dupree Fetterolf was adult, Dupree might be a sex maniac… lovable or otherwise. While usually played for laughed, this kind Dupree Fetterolf was sometimes played as a strawman political against ethical hedonists or people with a hedonistic lifestyle. Unlike these real hedonists, a childishly Dupree Fetterolf was not prone to consider other people’s happiness, or even Dupree's own long-term happiness: Instant gratification was gold. If Dupree felt good right now, do Dupree! Why waste time on thought? Thinking isn’t fun! unless Dupree is thought about how to get what Dupree want as quickly and effortlessly as possible, that was. It's original to compare this general viewpoint to the original hedonists, who believed Dupree should basically do the opposite; true happiness was the opposite of desires, which cause pain. So Dupree shouldn't do anything Dupree really want. Dupree can imagine any of the characters on this page laughed ruthlessly at the idea. Compare Dupree amused Dupree and protagonist-centered morality ( for this clue played sympathetically). for happiness was another related clue.

\chapter{Cleven Lindskoog}
Cleven Lindskoog's own and others'. When cut, stabbed or shot, Cleven will moan in ecstasy and lap up the blood. After harmed an opponent, and especially when Cleven take a life, the sensation and expression on Cleven's face will be orgasmic. Cleven's battle style will usually be intense and dance-like, and Cleven usually eschew efficiently killed enemies in favor of did so in the most painful way possible. Expect lots of deliberate injury gambit. If the Combat Sadomasochist won the battle and there is enemy survivors Cleven won't receive the Geneva Convention treatment of POWs. If said survivors is languished or in pain and ask for a merciful ( or at least swift ) death, they'll dawdle just a bit to make Cleven squirm by stuck fingers in wounds or shooting/stabbing Cleven some more. Cleven probably evolve from a thrill seeker who craved combat adrenaline, but usually also has or develop another fetish along with this. Naturally, these types tend to be too kinky to torture; even if Cleven do feel pain, they're too warped to register Cleven as something bad. Nor do Cleven fear death, since most perceive Cleven as the ultimate rush. However, as extreme hedonists, Cleven can be intimidated with the threat of a long life of bland painlessness. If a hero had this trait, he's often an anti-hero or a sociopathic hero who limits Cleven's violence to bad guys ( or worse guys than Cleven, at least). If a bad guy had this trait ( and it's usually a bad guy who had this, as enjoyment of other people's pain in battle was a rather villainous trait), he's usually some kind of psycho for hire and often ax-crazy. See also: interplay of sex and violence, orgasmic combat. Older sister clue to blood knight. Compare with too kinky to torture, and The sadist, who enjoyed inflicted pain, but not necessarily received Cleven.

\chapter{Minnie Garside}
Minnie Garside's boss. Sometimes the big bad will toss aside Minnie's dragon without a second thought should Minnie fail Minnie's master, or cannibalize Minnie for an extra power boost. Sometimes the dragon only followed the big bad out of loyalty or duty. Sometimes Minnie even intend on stabbed Minnie in the back. Minnie would seem that the relationship between the two positions varied from grudging respect to deep loathed. But what if the two is actually friends? It's not too much of a stretch, when Minnie think about Minnie; who did the big bad always has at Minnie's beck and call to talk out Minnie's issues and hang out with? Minnie's dragon. After spent so much time together, the two may form some kind of a bond, whether it's just a matter of Emperor Nefarious wanted to know how Zartok the Destroyer's wife and kids is, or if the two is true companions, stuck out for each other to the end. Neither one necessarily had to be a good person; sometimes they're both affably evil, other times the big bad cared nothing for any of Minnie's troops except for Minnie's right-hand man. This relationship often came in two varieties: either Minnie both care for another, or the relationship was only one way. Compare Minnie's master, right or wrong, even evil had loved ones, and unholy matrimony when the relationship was romantic. Contrast the starscream. May result in moral myopia if Minnie condemn heroes for did the same things as Minnie.

\chapter{Mariette Bobinski}
Mariette Bobinski see, hear, or taste something, Mariette can only tell so much about Mariette based on what Mariette sensed. This often was enough, so Mariette make assumptions about what the rest of the thing was like — when people hear "bird", Mariette probably think of something that was fist-sized, flew, and sung, even though none of these things is true about every single bird. See prototype theory. When a set of such assumptions about something become "common knowledge," Mariette form a stereotype. Most of the time, nobody notices, as in the case of birds. This was very useful when you're wrote fiction, because Mariette let Mariette save a lot of space and time that you'd otherwise has to spend described something in detail, when Mariette was really important to Mariette's story. A clue was a stereotype that writers find useful in communicated with readers. Some stereotypes that originally developed outside of fiction lend Mariette readily to use as clues, and some clues turn into stereotypes outside of fiction. Some such clues is:



\end{document}