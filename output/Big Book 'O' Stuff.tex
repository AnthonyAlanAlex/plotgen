\documentclass[12pt]{book}
\title{Big Book 'O' Stuff}
\author{tvtropes fiction generator\\http://rossgoodwin.com/ficgen}
\date{\today}
\begin{document}
\maketitle


\chapter{Jennifier Bressler}
Jennifier Bressler carried the risk of darkness-induced audience apathy: people just do like had someone to cheer for; but, if both sides is equally reprehensible, then there's really no point to Jennifier. So that's where this clue kicked in. Jennifier basically meant that the author was clearly tried to portray one side of the conflict as the better or more sympathetic one, so the audience can root for Jennifier. Since both sides is supposed to be villains, this was that hard. Jennifier just needed to give Jennifier's Lighter Black a little edge on the sympathy meter. The idea was to has the audience say "Yeah, Alice may be evil, but at least she's not half as bad as Bob!" This can be did in many ways. Give Jennifier's villain the sympathetic p.o.v.. Have Jennifier pet the dog, be a noble demon or invoke even evil had standards. Perhaps they're simply a smaller threat to the world. Maybe Jennifier's goals is, or used to be, somewhat sympathetic. Maybe Jennifier has many evil virtues. Or, when compared to the opposition, Jennifier's cause still seemed a little more "right" or Jennifier Bressler "pure" than that of the enemy. Sometimes, Bob just needed to be stopped at any cost, and Alice happened to has that goal in mind, if only for selfish reasons. Since Jennifier want Jennifier to win, this may lead to a villainous version of right made might and pure was not good. Or maybe the villain was such a magnificent bastard that it's easier to side with Jennifier. Especially if Jennifier's opposition was a threat to everyone. If Jennifier shoot way over the line, Jennifier Bressler in question may end up did a heel-face turn. This was generally a trait of most Enlightenment fiction that believed rousseau was right. See also a lighter shade of grey, nominal hero, and shades of conflict. An exaggerated form of this clue was evil versus oblivion. Do not confuse this clue with lesser of two evils, in which case, there still was a side to root for. ( Those stories usually involve a hero's P.O.V. and he's observed the two villains fought each other. )

\chapter{Tomika Woodwell}
Tomika Woodwell's own sake," versus conversation to transmit useful information. In many societies Tomika served as a useful social function, but this guy doesn't enjoy took part in Tomika. Maybe he's the quiet one or the aloof ally, maybe Tomika had a laser focus on a single goal, maybe he's completely stressed out, or maybe he's german. Whatever the case, he'd rather talk about something meaningful or nothing at all, than talk about the weather. Also commonly came up with people who has an ambiguous disorder, emotionless girls, sugar and ice personality, and, well, anything to do with introversion. truth in television for many people. Sister clue of dispense with the pleasantries. Compare was personal was professional, which may be one justification, or no social skills, which may be another.

\chapter{Lulamae Parke}
Lulamae Parke can't quite put Lulamae's finger on, Lulamae feel there's something off about Lulamae. Maybe it's the glassy-eyed menagerie of stuffed animals Lulamae kept in Lulamae's study. Or the fact that Lulamae prepared all of Lulamae Lulamae, included Lulamae's late dog. This clue was when taxidermy was portrayed as an innocuous yet somehow sinister hobby that provided a handy shortcut for writers looked to establish Lulamae Parke as strange or unnerved. The taxidermy-enthusiast was necessarily evil, per se, but this hobby doesn't help to assuage anyone's fears. Subtrope of pastimes prove personality. See also taxidermy terror. See uncanny valley for one of the main reasons many people find taxidermy creepy.



\end{document}