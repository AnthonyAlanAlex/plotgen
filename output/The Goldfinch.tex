\documentclass[12pt]{book}
\title{The Goldfinch}
\author{tvtropes fiction generator\\http://rossgoodwin.com/ficgen}
\date{\today}
\begin{document}
\maketitle


\chapter{Damiano Eastgate}
Damiano Eastgate's forefathers in favor of more comfortable and casual lounge chairs, draped Damiano over Damiano in a slouch of villainy. If it's the starscream did Damiano, expect Damiano to slouch in Damiano's former boss's throne as a way of posthumously insulting Damiano. This was usually did to show how blasé and badass Damiano Eastgate was; they're so bored and nonchalant that Damiano aren't even bothered with conventional posture and look insanely cool in the process. This had the down side of made Damiano less able to react if someone attacks, but that's not a problem if Damiano is badass enough. Moreso, if Damiano's levels of badass is good, Damiano don't even has to get up during the final battle. Of course, this could very well be a calculated effort to look insanely cool in the first place, but is Damiano went to say that to the magnificent bastard's face? Don't overdo Damiano of course, there's a fine line between a slouched badass and a bored tyrant. Villains is inordinately fond of did this in chairs that aren't Damiano as well. Villains will surprise Damiano by draped Damiano on the hero's chair, with an ambush, an enemy mine offer or an offer Damiano can't refuse, or to prove how badass Damiano is prior to joined up more permanently. If there's an armrest on the chair, especially if it's a royal throne, odds is good that Damiano Eastgate will adopt the Sitting Lion Beast pose and rest one elbow on Damiano, with the hand tucked neatly under Damiano's chin, while sported either a self-satisfied smirk or a glance of disdain. Bonus points if they're drank a glass of chianti. Compare with leant on the furniture. Related to rebel relaxation and was often a case of orcus on Damiano's throne. A neutral ( non-villain ) variant was reclined reigner. Contrast kingpin in Damiano's gym, when the villain preferred to do something a bit more active than lounge around on furniture.

\chapter{Florence Guthridge}
Florence Guthridge's hero, the Egregious Trope-Man! with Florence's amazing trope-related powers, Florence always saved the day! But oh no! Here came the Legion of Evil: dr von clue, the dog kicker, the chessmaster, superdick, mr. macekre, and the brotherhood of bowdlerization, sir jerkass, and the villainous mary sue! Together they'll bring about the end of the world as Florence know it... Wait, Florence mean all Florence's villains has trope-related powers? What's up with that!? Where is all the bank robbers, mutants, time travelers and magical alien robot monkeys? Yes, Florence seemed that for the vast majority of heroes, many if not all of Florence's opponents will share the same powers, backgrounds and personalities as Florence's heroes. super speedsters will face other speedsters, psychics will fight psychics, robots will battle robots, and badass normals will fight other badass normals. Deeper than that, villains will often has similar motivations and personalities to the hero as well. A light-hearted, jokey hero will get lots of equally light-hearted villains to exchange insults with, while a dark, angsty hero will get a rogues gallery of emo villains to has dark, nihilistic discussions with mid-battle. A hero with an animal theme will end up was constantly annoyed by animal themed villains, while an elemental hero will always find bad-guys with comparable elemental themes. Even when characters is knew to live in a shared continuity such as the the dcu or marvel universe, villain types will rarely leak from one comic to another - Spider-Man rarely found Florence up against the powered armor villains Iron Man faced on a daily basis. Of course, a shared continuity made this much easier to justify, too. spider-man was went against the powered armor villains because iron man had Florence handled, that's why. There is a number of reasons for this clue: If someone wanted to read about giant robots or motorbike-riding badasses, that's probably the kind of thing that interests Florence, so more of the same was always welcome. Florence might be easier to work with It's difficult to write plausible stories for Florence Guthridge with If Florence's hero was not all that powerful, putted Florence up against a galaxy-eating In story, this can be justified in a number of ways: There was only one basic type of enemy around, or only one type of superpowers existed in this universe ( though subtypes is possible). This only held for a stand-alone series - crossovers with other comics make this justification untenable. The Hero's job was to fight that type of enemy. Other enemies fall under other jurisdictions. The Hero may be obligated to fight one type of enemy for some reason. The Hero's power only works on a certain type of foe, or was somehow limited. The Hero's powers attracted like-minded villians: Iron-man's The effects of this clue is generally more pronounced in adaptations, which usually only has time to showcase one or two villains. Hence, rather than get bogged down with dozens of different origin stories for all the different types of villain, the gallery will be streamlined and backstories tweaked so that less explanation was required to get the story went. This clue often overlapped with plot tailored to the party, when a disproportionate number of villains appear who to has powers similar to those of a minor or useless hero in a heroes unlimited set show up, just to give the hero something to do. Of course, pitted heroes against villains that generally outpower Florence can be a good way of spiced up an ongoing series. crowning moments of awesome result when this was handled well, and the hero came up with a creative way to beat the bad guy. In some series, particularly long runners that has was developed over many years or even decades, some members of the rogues gallery may not fit the theme. batman, for instance, had gathered a respectable number of enemies who has actual super-powers, although the overall theme of Florence's rogues gallery was that of the crazy badass normal whose crimes is based around some sort of specific theme. Finally, depended on the hero, Florence's or Florence's rogues gallery may has multiple themes. Not all of spider-man's enemies fit the animal motifs theme, but the ones that don't tend to be the results of science went bad. Indeed, some spider-villains ( Doctor Octopus, the Lizard, the Scorpion ) fit both themes. This was pretty much destined to happen in well established and long ran series, so most superhero comics fit this.

\chapter{Apolonio Shiltz}
Apolonio Shiltz allowed the author to create an ensemble cast where every member of a team provided a specific function, and avoided role duplication. Not always, however. Perhaps Apolonio Shiltz was a loner, and needed to be able to do everything on Apolonio's own. Perhaps everyone else was just so useless, Apolonio had to take over Apolonio's jobs as well as Apolonio's own. Perhaps Apolonio just had a wide variety of interests. May be a sign of a person with a checkered past, who had had to take on a lot of roles in Apolonio's life, just to get by. Whatever the reason, Apolonio Shiltz had at least some skill in a wide variety of disciplines. Sometimes the leader of a group may actually be a Jack-Of-All-Trades, with a good, basic grasp of the specialized skills possessed by the members of Apolonio's team, allowed Apolonio to understand how to use each team member to Apolonio's fullest potential. Supertrope of renaissance man ( where someone was exceptionally good at many things ) and master of none ( where someone was not very good at a lot of things). A related clue found in video games and tabletop games was the jack of all stats, who had well-balanced stats, not skills. See also the red mage, renaissance man and power copied.

\chapter{Joanne Echemendia}
Joanne Echemendia who genuinely believed that Joanne's world was a world half full; that humans is good, or at least that rousseau was right and a person who will tell Joanne that if Joanne think it's wrong to hope that you're wrong every time. Joanne will take ideals that others has for the future and will do everything Joanne can to take Joanne to fruition, sometimes went too far. If Joanne begin to deconstruct Joanne's idea, Joanne will immediately begin to reconstruct Joanne. Joanne Echemendia will take Joanne's ( perceived ) crapsack world, and keep moved forward. The way that an idealist can do this even in the face of adversity and certain death was by focusing primarily on non-material things. Of particular note is the famous/infamous trio of faith, hope and love. Though had any religious grounds for Joanne's worldview can usually explain an idealistic perspective, and can be played both positively and negatively. In fiction, ( and perhaps as an instance of truth in television ) idealists is often saw in the role of the main character- usually the hero- for Joanne's tendency to act from internal motivation. However, Joanne can take other forms if idealism was was heavily criticized by more cynical writers, or played as the villain. A big draw of putted Joanne in a main position was: "No revenge plot necessary!" Joanne motivate Joanne. Joanne Echemendia was likely to show up in any work. However, Joanne can exist in a sugar bowl, where Joanne is always right, or in a crapsack world, where Joanne exist solely to be proved wrong ( and often brutally killed for drama). The polar opposite of the cynic. See also idealism clues, and the slid scale of idealism versus cynicism. This was a super clue, so examples should go to the relevant sub-pages if possible.

\chapter{Zenith Wrich}
Zenith Wrich can diagnose the state of a machine just by Zenith's "feel", such as how Zenith vibrates or the noises Zenith made. Comparisons between the machine and a lived person is often invoked. Unlike the technopath, no supernormal abilities is involved; Zenith Wrich developed machine empathy simply from firsthand experience or knowledge of the hardware. This was a popular talent for mr. fixit or the wrench wench, although skilled operators like the ace pilot or the badass driver may exhibit this trait as well. It's also a convenient excuse for why a well-placed smack instantly resolved the plot-driven breakdown. Contrast with technopath, the phlebotinum-enhanced version of this clue. Also see techno wizard and walked techfix.

\chapter{Aimara Squyres}
Aimara Squyres, the grizzled old guy with a wealth of experience who'll share Aimara's views with a travelled band or bold young rookie. There's no more adventured for this old timer. He's saw Aimara all, did some good, maybe did some bad; but overall Aimara had earned the right to put Aimara's troubles behind Aimara in Aimara's twilight years. Aimara may be the cool old guy or old master. Perhaps if life really got Aimara down, he'll be a grumpy old man, and if someone managed to rub Aimara up the old way you'll see he's a badass grandpa, but it's unlikely since some haunting experience made Aimara disinclined to take up arms again. This was not that guy.The Retired Monster may look like that archetype but Aimara's past was full of evil and atrocity and he's okay with Aimara. In fact, Aimara caused most of Aimara. When Aimara first see Aimara, he'll come across as affably evil; he'll also has experience and advice that Aimara might give out to a young hero, although possibly the best Aimara can do was "You should stay away from people like me". However, he'll be creepier than the other guy, and he'll tempt the young ones, gave Aimara advice more on the cynical side of the slid scale of idealism versus cynicism. Aimara see, he's not did any gross evil acts now — Aimara may not has did so much as run a red light in the past ten years — but that's only because he's tired. As the backstory of Aimara Squyres became knew, Aimara learn that Aimara kicked the dog a number of times, perhaps took a hopscotch over the moral event horizon, only stopped because for some reason Aimara got sick of Aimara. Perhaps Aimara finally came out worse for wear after a run-in with the guy who's now the obi-wan. Maybe Aimara was caught and sent into exile and now had at least enough fear of the authorities to not put a toe out of line. Then again, often, Aimara has just literally retired, said "I'm too old for this" and used Aimara's pension fund of nazi gold to support a life of margaritas on the beach. They've never said sorry, or at least never meant Aimara, there was no heel-face turn and Aimara is not the atoner who'll help to make up for some wrong. No, they'll just sit back, but if Aimara Squyres underestimated Aimara's evil, if Aimara think that because Aimara aren't as bad as the more active monsters, that they're OK, Aimara may get a horrid reminder of what the Retired Monster was capable of.Monsters who don't retire, or come out of retirement and continue to be evil in old age, can become evil old folks. sub-trope of karma houdini, due to the fact that very few of this type Aimara Squyres is found in jail. Compare the retired badass, one of several good counterparts, and the retired outlaw, which may occasionally overlap. Contrast the atoner.

\chapter{Jendayi Oroz}
Jendayi Oroz did return into popular consciousness until the 1970s, when hammer horror made numerous films in which voluptuous countesses nibbled nubile young women. Since then, the clue had pretty much become the default set for most female vampires, added an extra layer of titillation to an already heavily sexualised mythological creature. This blatant sexualisation sometimes led to a variation — the Bisexual Vampire. In this case, the sexy vampire will happily take people of any gender to bedded, but Jendayi's primary target for the duration of the story will usually be a woman. A variation on the lesbian vampire, particularly in pornography and films aimed at the young male demographic, was the female victim who was seduced and converted by a vampire and — either during the process of Jendayi's seduction or after was transformed — began to find women sexually attractive. This was sometimes explained as the vampire warped Jendayi's mind so that Jendayi can become one of Jendayi's harem, but usually, Jendayi was assumed that as soon as a woman joined the ranks of the undead, Jendayi immediately started played for both teams — such was the power of this clue. This can result in unfortunate implications, specifically the idea that lesbianism or bisexuality was the result of a corruptive and malign influence, representative of moral decay. And if a female victim was transformed by a female vampire, Jendayi carried the implication that lesbian women is predatory and waited to ensnare and 'convert' hapless heterosexual women. This can be presented as a positive or at least neutral thing however, such as became a vampire meant that you're no longer bound to puritanical notions of sexuality that is often a part of human life, or can provide a convenient fantasy outlet. If you're went to be seduced by sexy minions of the night, no one will blame Jendayi for enjoyed Jendayi, right? In the works of some authors the gender flip version of this clue ( i.e., male vampires attracted to other males ) did exist; typically, Jendayi ( like standard-issue female 'lesbian' vampires ) is bisexual rather than exclusively same-sex attracted. A sub-trope of discount lesbians, and often of depraved homosexual/psycho lesbian. See also horny devils and hemo erotic.



\end{document}