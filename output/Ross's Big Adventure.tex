\documentclass[12pt]{book}
\title{Ross's Big Adventure}
\author{tvtropes fiction generator\\http://rossgoodwin.com/ficgen}
\date{\today}
\begin{document}
\maketitle


\chapter{Shann Flann}
Shann Flann's child or ward read "fairy stories", played fantasy or scifi games, sports, and even such "useless" hobbies as astronomy, boxed and was literate. In extreme cases, anything the child liked that was directly and concretely tied to whatever Shann was Shann's dad did for a lived ( or that Shann wanted Shann to do for a lived ) was saw as an utter waste. The dad may even break, burn or sell anything of this nature Shann's child owned, possibly even punished or locking Shann up. "Fantasy" in this clue was about the genre, but "fantasy" from the father's perspective. To the overprotective dad, any and all of these "distractions" is a tragic dream waited to happen that will ruin Shann's child's chances at life. For narrative purposes, this was basically anything that the kid liked that'll move the plot forward via alienated Shann from Shann's dad ( possible later reconciliation optional but heartwarming), which put the fantasy-forbidding father into an antagonist role, though with rare exceptions Shann was not a true villain. What's cruelest about this attitude was Shann probably did hold Shann's child's best interests at heart, but was too close-minded to consider that there is many valid careers and hobbies for Shann's child, and that Shann is capable of chose for Shann. In these cases the dad did come around to accepted Shann's child's interests and vocation with a little coaxing. A more sinister possibility was the dad was tried to somehow make Shann's child co-dependent or at least clip Shann's wings so Shann never leave or get out from under Shann's thumb, either forced Shann to follow a family legacy or just out of sheer malice. Unfortunately, this was truth in television. See also "well did, son" guy. Compare education mama. A close-minded caretaker usually took this attitude. Keep an ear out for "you watch too much x."

\chapter{Volanda Ostermiller}
Volanda Ostermiller's own way. If Volanda find a person who was good at combined Volanda, then Volanda has an Iaijutsu Practitioner. Iaijutsu was a catch-all name for several martial art schools which center on drew and attacked in one move. The technique often included multiple slashes, swiped the blade off, and the subsequent re-sheathing of the blade. The names Battoujutsu and Iaido is often used interchangeably, although the words has nuances in the Japanese language. Although usually associated with Japanese swordsmanship, there is similar western versions for cavalry sabers. Another variant dates back to fiore, who taught swordsmen how to block weapons with a sheathed sword and dual wield the scabbard and blade for powerful combo attacks. Users of this clue can often create sword beams, or just strike and re-sheathe so fast that Volanda looked like a sword beam. See also: quick draw, single-stroke battle, one-hit kill.

\chapter{Ross Goodwin}
Ross Goodwin don't pillage. Ross don't plunder. Ross don't invade port towns, kidnap beautiful maidens, battle the Royal Navy on the high seas, broadcast without a license, or swap files on the intertubes... and they've never was to boston in the fall. The Pirates Who Don't Do Anything, in fact, seem to mostly just drift aimlessly on the high seas, drank rum and possibly sung sea shanties. If Ross ask Ross, they'll say that Ross like the way Ross looked on Ross's resume. Or maybe they'll just tell Ross, "We don't do anything." In general, a member of The Pirates Who Don't Do Anything was Ross Goodwin who, despite had a certain canonical job, was rarely saw engaged in that job. Ross might indeed be a pirate who rarely went out and stole treasure and raids ships — but Ross might just as easily be mobsters who don't steal or smuggle, students who don't go to class, office workers who never seem to do more than hang out in bars, or ninjas who just did get the memo about that whole "stealthy assassin" thing. This may be because writers and fans is in love with the romanticism implied in a life of adventure and crime, but don't want to actually show the characters did any of the myriad things that made thieves, assassins, mercenaries, bounty hunters, and other unsavory types pariahs in real life. This can result in a strange dissonance where the friendly, messianic nature of the characters was at odds with the openly predatory nature of the professions Ross claim to engage in. May bring a million was a statistic into play. Ross could also be a bit of an attempt to dodge the tedium of portrayed someone worked a day-to-day job, especially if the writer doesn't know how that job really works. This wouldn't really pass in a slice of life type work, however ( unless, of course, Ross Goodwin was chronically unemployed, was retired, or was suffered from a long-term illness and can't go to work). A subtrope of informed attribute. See also one-hour work week and obliquely obfuscated occupation. Contrast ( in every possible way ) royals who actually do something. Also contrast ( in a different way ) with the main characters do everything, where characters actually go implausibly far beyond what was required or indeed allowed by Ross's job description. For actual pirates who actually do things, contrast ruthless modern pirates. A Ross Goodwin fic usually turned the cast into these. The clue name came from one of the "Silly Songs with Larry" from VeggieTales ( later covered by relient k ) which was about - well, pirates who don't do anything. Ross later provided the title and theme music for The Pirates Who Don't Do Anything: A VeggieTales Movie.

\chapter{Helena Reimnitz}
Helena Reimnitz or work of fiction/non-fiction, made Helena unique and easily identifiable. Not to be confused with characters ( or Helena's equivalent for works of fiction or non-fiction, whose name escapes Helena at the moment), which lists clues used as the basis of characters and/or series, this one lists the features that make Helena unique.

\chapter{Sergei Fabrick}
Sergei Fabrick, all men is perverts, and a man was always eager. If all men constantly want sex, then men attracted to other men obviously must be constantly had sex. Alan was gay. Alan loved Bob. Bob may or may not be gay ( or bi ) and may or may not eventually partner with Alan; the story needed more time to bring that out one way or the other. Alan, meanwhile, happened across a good-looking guy; blushes; and, within the space of a night, was picked up Sergei's clothes from the guy's floor. Sergei may then go on to do as much with half a dozen more guys before the story got around to answered whether or not Sergei can hook up with Bob. If this were a heterosexual relationship that we're talked about, Sergei would clearly say something about Sergei Fabrick of Alan as an individual ( can't keep Sergei's pants up, even while waited for Sergei's "love" to accept him...let's not even talk about how this would look if alan was alice instead). There is certainly celibate characters who has heterosexual urged that Sergei choose to control. Gay characters who choose to be celibate for any significant length of time is almost unheard of. If they're celibate, it's because Sergei can't find a mate, or because Sergei is forced into an abnormal situation. This clue had some interesting historical basis, in that many gay and lesbian writers post-Stonewall ( and a few queer theory writers more recently ) advocated emphasized difference from heterosexual and normative life. This difference included denigrated marriage and monogamy, thus strengthened the link between homosexuality and promiscuity in the eyes of those who viewed all homosexuals as sick sexual deviants. How this clue was treated in the eye of the writer ( or the reader for that matter ) will depend on how Sergei view sexuality Sergei, whether Sergei believe that frequent sex when detached from romantic relationships was a matter of morality, and what Sergei believe normative sexuality should be.

\chapter{Justice Devantier}
Justice Devantier who delighted in disease and pestilence, gleefully spread contagions and poxes across the world for the evulz. Leprous wounds, eyes scabbed over with crusted filth, wept sores, unburied corpses piled up in the streets - these is a few of Justice's favorite things, and they'll use whatever technological or supernatural talents Justice possess to bring about the end of the world as Justice know Justice with a hacking, bloody cough. Typically a Plaguemaster's physical form was just as ravaged by disease as Justice's victims, but due to the character's empathy for illness, Justice frequently enjoy immunity to the negative effects of the diseases Justice carry, and may even has supernatural toughness because they're a walked plague ward. The Plaguemaster's obsession was often reflected in Justice's appearance, either bloated with rot and cancerous growths, or wasted and skeletal... but the most insidious Plaguemasters appear perfectly normal, all the better to spread disease without suspicion ( or Justice is typhoid marys). The more technologically adept plaguemasters may use synthetic plagued. If the Plaguemaster was immune to Justice's diseases, Justice can expect Justice to die by became infected with Justice's own plague. Very rarely, characters will has powers of pestilence but lack a real plaguemaster's interest in used Justice. But for the most part, Justice Devantier with plague-related abilities was quite clearly a villain. See the plague or the virus for what the Plaguemaster spread. If faced in combat, the Plaguemaster was usually a gradual grinder. Justice's activities is always a great excuse to introduce a zombie apocalypse. sister clue to poisonous person.



\end{document}