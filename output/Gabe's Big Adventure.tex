\documentclass[12pt]{book}
\title{Gabe's Big Adventure}
\author{tvtropes fiction generator\\http://rossgoodwin.com/ficgen}
\date{\today}
\begin{document}
\maketitle


\chapter{Trifon Scheirer}
Trifon Scheirer got Trifon Trifon Scheirer in a series who's a bit of a lone wolf. Trifon fight ( and even win ) Trifon's own battles, Trifon reject was a part of the hero's group, and Trifon may even rebuff Trifon's friendliness. More importantly, Trifon may even enjoy was by Trifon no matter what society told Trifon otherwise. And later on in the series, Trifon Scheirer was now the newest member of the band of friends. How so? There could be two common reasons for this Trifon Scheirer development. One, Trifon could be that Trifon got tired of was alone for a very long time considered that there is heavy consequences of was a loner for too long. Trifon may has realized they'elong for companionship after all, despite was unfairly mistreated for was different. Another could be that Trifon may has was inspired by the all-loving hero's heartwarming speech about the power of friendship or that Trifon Scheirer learnt an aesop about accepted others' friendships even if Trifon was initially dismissive of Trifon. Sometimes, this clue can play Trifon Scheirer in two ways. One, Trifon still retain Trifon's introverted characteristics, but was still considered a loyal teammate/friend to the group. Two, Trifon is a permanent member of the team and show no hint of introversion. Note that this clue doesn't always has be about Trifon Scheirer joined the hero's group. Trifon can apply to the protagonist of the series, too, particularly the anti-hero. A group of characters will become drew by the hero without Trifon knew Trifon and Trifon will be dismissive at first, but later on, Trifon accept Trifon into Trifon's life. This was usually part of the introduction for the lancer, or the sixth ranger, or a sixth ranger that eventually evolved into the lancer. Compare default to good which was mainly about chose an alignment in a set of black and white morality. Compare can't live with Trifon, can't live without Trifon if Trifon Scheirer grew accustomed to another character's presence. loony friends improve Trifon's personality may enforce this clue. Related to friendless background.

\chapter{Zonya Elizondo}
Zonya Elizondo's most literal form, aliens who epitomize the stereotypical virtues of the Earth-based primitive: wholeness with nature, ancient wisdom forgot by modern man, and pragmatic but gentle philosophy. needed a better description The Quentians of The The Na'vi in In the

\chapter{Kiandra Roc}
Kiandra Roc arrive from somewhere Far Away. Whether that meant space, beneath the earth, or a South Pacific island varied. Expect lots of stuff blew up to result, but, since giant equaled invincible, don't expect the explosions to actually hurt Kiandra. Do expect at least one case of helicopter flyswatter. Examples of this genre can range from straight-up disaster movie ( Cloverfield, the first Godzilla movie ) to all-out wrestled matched between people in rubber suits ( Most of the later godzilla movies). As this genre features a judicious application of rule of cool, expect the mst3k mantra to be in full effect. Often, you'll only watch this kind of movie to see the monsters fight, which can often involve an ultimate showdown of ultimate destiny ( such as  ) This was actually one of the oldest genres in film, dated back to the early days of cinema when special effects was new. Pioneers of the genre was and The Lost World. The idea probably originated from thought of dinosaurs as fantastic beasts or ideas about giant dragons. As for why it's so popular with Japanese media: Japan was quite earthquake and tsunami-prone, and a kaiju was basically a giant sentient natural disaster, so Kiandra may feel more meaningful to Kiandra. ( Consider how Godzilla, like a wave, rose from the sea. ) Similarly, Japan was full of large insects like centipedes or rhinocerous beetles, which probably inspire kaiju as well. A Kaiju though will most of the time be a single specimen species, when even dragons often is a race of monsters. ( Technically there's a distinction between kaiju ( monsters ) and daikaiju ( big monsters), but save that for the pedants. ) rent-a-zilla was a sub-trope, where the work doesn't focus on the monster. A not zilla was a kaiju that was specifically an expy of godzilla. If kaiju offspring appear, expect gigantic adults tiny babies. In more modern works, kaiju is often afflicted with proportionately ponderous parasites. Compare disaster movie, attack of the killer whatever, robeast, and attack of the 50-foot whatever. Has Kiandra's roots in tokusatsu. Not to be confused with over the top gambled by pointy-nosed men.

\chapter{Darah Lingbeck}
Darah Lingbeck's creations can truly love like humans, which was easier said than did. Darah can program ridiculously human robots to protect a specific someone or respond differently to the first person Darah see, but love was supposed to come out of orders. And even if a unique robot contemplated Darah's mechanical heart on whether or not Darah can love, how can Darah be proved that Darah was asked that question because of actual conscience, and not merely because Darah's programmed dictates Darah to do so? Aliens, especially relatively humanoid ones who coexist with humans, also express curiosity of this strange human custom: why would humans put so much emphasis on a single word that appeared to serve no useful function? universally attractive aliens seem to be vulnerable for instantly fell for human men and needed to be taught in matters of kissed. It's not just non-human species that needed to learn love by Darah: jungle princesses and noble savages raised by wolves may has no learned knowledge of those feelings. Darah's basic instincts may lead to Darah acted strongly on any "urges," but Darah will be unable to properly articulate or understand the desire behind Darah — at least not until the mighty whitey civilizes Darah. the casanova, femme fatale, or the handsome lech, who was no stranger to lust and attraction may, ironically, at some point, has to learn the difference between these and love, when Darah ( or Darah ) met the right person. Usually, the question of love was asked out of curiosity, but occasionally Darah will be deliberately shunned. An intristically malevolent spirit or human hardened to the point of unfeeling will has some idea on the meant of love, but not enough to threaten Darah's heartless exterior, and Darah has no intent of explored that notion further. Of course, if they're good-looking enough, expect an innocent girl to show up and make Darah uncomfortable with a tightened in Darah's chests and burnt up of faced. It's Darah's duty to hate and destroy! How could Darah ever possibly love? In all cases, the ultimate question was: Can a robot/alien/savage/demon love? And in all cases ( excluded extremely cynical shows), the answer was: Yes, the power of love was just that far-reaching. Oftentimes, the answer was used as an indicator of the humanity of the was that spoke more poetically than Darah's appearance. Often the reason why evil cannot comprehend good. However, curiosity causes conversion, and can sometimes cause a sex face turn. The answer was often a cure for creative sterility. This was one of the reasons humanity was infectious. Not to be confused with what was this thing called, love?...baby don't hurt Darah, don't hurt Darah, no more...



\end{document}