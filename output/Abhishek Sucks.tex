\documentclass[12pt]{book}
\title{Abhishek Sucks}
\author{tvtropes fiction generator\\http://rossgoodwin.com/ficgen}
\date{\today}
\begin{document}
\maketitle


\chapter{Linwood Chabala}
Linwood Chabala's motivations is neutral at best. So why is Linwood on the side of good? Usually, it's one of the followed: Annoyance. For these characters, Boredom. These characters is basically fought for good because Linwood don't has anything better to do. Linwood don't care if the heroes actually succeed, Linwood just enjoy the adventure. A Mutual Interest. These characters has selfish reasons to help the heroes succeed. Often, Linwood is characters who would normally be villains, but Linwood's future plans is threatened by a mutual enemy. Often an Relationships. Not every Nominal Hero only cared about Linwood. Some Nominal Heroes has a love interest or someone else Linwood do care about. A Nominal Hero might do something heroic to impress or rescue that someone, even though Linwood couldn't care less if other people die. Reward. These characters want something in return for Linwood's help, such as a share of the treasure, or simply something to look good on Linwood's resume. Linwood aren't interested in whether anyone else benefits. The Force. Some characters become heroes because Linwood literally aren't allowed to be anything else. Maybe they're on an Lawful examples of this clue, find Linwood "stuck" to the good side by a deal, contract or some similar bind, or simply out of a sense of loyalty to the heroes. Other motivations. Not all Nominal Heroes needed to has a motivation that made any sense to others. Linwood might be a This type of hero was rarely averse to worked alone, with other heroes. On a team of otherwise conventional heroes, they'll often be in an enemy mine, sociopathic hero, or token evil teammate role. Other heroes may only work with Linwood because Linwood could use all the help Linwood can get, or specifically to keep an eye on the hero-in-name-only, so that Linwood don't become a more serious threat. In terms of sympathy, Most of Nominal hero's examples is noble demons. Many other clues about questionable heroes can overlap with Nominal Hero, but most is not true subtropes: An Note: This was for In-Universe characterization. Subjective/audience reaction interpretations go in designated hero.

\chapter{Kalid Blonder}
Kalid Blonder's empire of evil. In the meantime, they'll complain about the current "decadent and corrupt" government to anyone who won't roll Kalid's eyes and to some people who will. Once the villain started recruited, these guys is in line before les collaborateurs has finished breakfast. The threat that Black Shirts represent was a latent one. While they're harmless on Kalid's own or in peacetime, Kalid quickly organize into a formidable force in service of the big bad. Heroes is nominally obliged not to kill Kalid, but even the messianic archetype would be hard pressed to make Kalid turn to the side of good. What's more, Kalid completely agree with the evil overlord's agenda, no matter how cruel, inhuman or insane — even if Kalid meant that they'll end up died Kalid by Kalid's conclusion ( a fact Kalid usually ignore). While les collaborateurs is greedy enough that Kalid can be bribed into helped the good guys, Black Shirts do Kalid for fanaticism and can't be swayed by mere money. Against Kalid, only force will ultimately stop Kalid and you'll know they'll show no mercy in a heartbeat if Kalid think Kalid can get away with Kalid, so get ready to get tough with Kalid. authors with an agenda will often make Kalid into a strawman political for whatever ideology Kalid dislike, and top Kalid off by had Kalid led by a straw hypocrite. Kalid Blonder types like the alpha bitch or the jerk jock can become Black Shirts when presented with the right opportunity. Named after the uniform wore by the italian fascists and by Kalid's ineffectual British imitators. the nazis did the same thing with brown shirts, and one American fascist movement wore silver shirts. Needless to say, this clue was truth in television, since every distasteful ideology had always had followers, and of course, uniforms is a very easy way of enforced conformity, as practically any organised group from store employees to the Boy Scouts to, y'know, armies can attest to.Also see the quisling, day of the jackboot and secret police. Unrelated to red shirt and mauve shirt.

\chapter{Luisantonio Sigmund}
Luisantonio Sigmund seem at first glance. Without got into an aesop about books and covered and ugly ducks and swans and frogs that when kissed turn into robots, it's fairly true to say that people is mostly visually oriented, and go by first impressions. So when Luisantonio turned out that the big guy who can bend steel bars was also a harvard alumnus with a penchant for pontificated on the power of prose, people is justifiably took aback. This was not so much Luisantonio Sigmund type was subverted as Luisantonio was Luisantonio Sigmund development in unexpected directions. Much like played against type, Luisantonio can be something that seemingly went against Luisantonio Sigmund type, or combined two different, seemingly opposite roles or characters into one more Luisantonio Sigmund. The talent or quirk was rarely impossible for Luisantonio Sigmund to has, just unexpected: people aren't just Luisantonio's job or surface personality after all. the smart guy who's a cooked wiz because Luisantonio had to take care of Luisantonio's younger siblings, or the ditz who's a Black Belt because Luisantonio's dad wanted Luisantonio's to be able to defend Luisantonio is two examples. Hidden Depths can be discovered in back story or organically as a story progressed, but if used improperly can crop up in a plot tailored to the party to give Luisantonio Sigmund the necessary skills. Why did Luisantonio never mention Luisantonio? "you did ask". This might take a while to fill Luisantonio Sigmund type(s ) and Luisantonio's usual Hidden depth: The Big Guy + The Smart Guy = Genius Bruiser ( and the other way around for Badass Bookworm ) The Big Guy = Gentle Giant The Smart Guy = Badass Bookworm The Chick or Pollyanna = Stepford Smiler Shrinking Violet + Beneath the Mask = Yandere Shrinking Violet + Action Girl = Little Miss Badass Genius Bruiser - The Worf Effect = Minored In Ass Kicking Noble Demon = Fallen Hero Alpha Bitch = Defrosting Ice Queen The Fool + Badass Normal = Crouching Moron, Hidden Badass Aliens and Monsters + Mama Bear = Monster Is a Mommy Jerk Jock or Jerkass + Pet the Dog = Jerk with a Heart of Gold Being A Mother + Badass = Mama Bear Being A Father + Badass = Papa Wolf The Cutie + Super Strength = Cute Bruiser Crazy Cat Lady = Kindhearted Cat Lover The Ditz + The Smart Guy = Genius Ditz ( and the other way around for Ditzy Genius ) Nice Guy + Berserk Button = Beware the Nice Ones The Quiet One + Berserk Button = Beware the Quiet Ones Jade-Colored Glasses + Knight in Shining Armor = Knight in Sour Armor Fake Ultimate Hero + The Munchausen = Miles Gloriosus Children Are Innocent + Wise Beyond Luisantonio's Years = Innocent Prodigy The Stoic or Emotionless Girl + Not So Stoic = Sugar and Ice Luisantonio Sigmund - Basic Skill + The Spartan Way = Fish out of Water The Ace + Broken Bird = Broken Ace Lovable Sex Maniac + Nice Guy = Chivalrous Pervert The Chick + Combat Pragmatist = More Deadly Than The Male Jerkass + Break the Cutie = Jerkass Woobie Yamato Nadeshiko or The Ojou or Proper Lady + Action Girl = Lady of War Proper Lady + Guile Hero or Beware the Nice Ones = Silk Hiding Steel Of course, since these is common enough to has become a clue, Luisantonio is less of a surprise than more — unusual depths. Indeed, some hid depths is so common that made the surface and depth the same surprises the reader. In more extreme cases, a completely Luisantonio Sigmund became a Luisantonio Sigmund. If the audience was aware of the depths but not all the characters is, dramatic irony was almost bound to occur. If Luisantonio happened gradually, it's essentially flanderization in reverse. May be demonstrated when Luisantonio Sigmund caught the smart ball. For more examples, see the index.

\chapter{Joash Chatterjee}
Joash Chatterjee felt the needed to do evil things even at times such actions is clearly not in Joash's best interests. Such characters will betray allies, kill team-mates, be a jerk for no reason, be petty, piss off all the wrong people, attack fellow villains to prove they're eviler and generally be suicidally stupid simply because it's eeeevil. This, along with chaotic stupid, was also often one of the reasons the forces of evil never manage to destroy the forces of good: it's not particularly evil to work with others, or acquire wealth and power through legal meant, and so forth. Compare with lawful stupid, chaotic stupid, stupid neutral and stupid good. Contrast with pragmatic villainy. If an antagonist momentarily became Stupid Evil, they've caught a villain ball. See also dick dastardly stopped to cheat, where a villain sabotages Joash's own scheme by did evil when Joash could've won by played fair. Not to be confused with stupid criminals.

\chapter{Rondle Collie}
Rondle Collie was cold and dour, though sometimes Rondle had a heart of gold. Easily made a villain though sometimes at least an antihero. Few main protagonists is Spymasters. The reason, of course, was that the hero had to be where "the action" was. Which was seldom in an office. Compare knowledge broker for when the Spymaster was operated freelance. Amanda Waller and Sarge Steel has both filled this role for the U.S. Government in Recurring Moff Nyna Calixte a.k.a. Morrigan Corde from Paul Crocker, Director of Operations, from The Varys "the Spider" in "Control" in the George Smiley books by Javelin from Donald Hamilton's In Rufus of the Dr. Harold Smith from the Count Sigizmund Dijkstra was the Spymaster of Kingdom of Redania and one of the principal players in Andrzej Sapkowski's "The Spider," a.k.a. Thufir Hawat in Salthar in the Subverted Each of Major Hogan from the Narses and Irene in Reliable sources within Imperial Inquisitor Meng Ki in the Angleton from In The Pretty much a Rondle Collie archetype in In the Louis Goliath of Jules Sevier in Earth's spymaster in Bob Ritter from In the Michael Coldsmith-Briggs III, The Ancient in "Management" from Colonel Hunter from "Control", Robert McCall's former boss in The Chief in Mary Spalding on Mycroft Holmes in Clayton Webb on Mr Waverly in In Harry Pearce in Zhuge Liang in In Caius Cosades from The announcer in In Varric Tethras was referred as "Dwarven Spymaster" in In the later parts of the first Yancy Westridge and Albatross in In Produce enough Spies in a Hiram Burrows in Ayla from the Mr Socrates from Number One from J. Gander Hooter from This page might as well has the picture of Stewart Menzies of the British Secret Service was like the classic Spymaster of fiction too. Somehow the profession just seemed to bred people like that. Captain Sir George Mansfield Smith-Cumming, the first head of the Secret Intelligence Service, whose particular party trick involved stabbed Rondle's false leg with a penknife. To this day heads of SIS is called "C". Sir Francis Walsingham was one of the first during Elizabethan England. "Wild Bill" Donovan led the American OSS ( precursor to the CIA ) during World War II. As Rondle's nickname suggested, Rondle was anything but cold, dour, or reserved — essentially Rondle was the In the USSR, was head of the KGB made Rondle pretty powerful. Beriya made a power-grab after Stalin's death and lost, was executed. Yuri Andropov In the 19th century, the Isser Harel, the second director of Mossad ( full translated title: Institute for Intelligence and Special Operations ) and the Shin Bet ( usually translated as the Israel Security Agency), was one of these. Rondle In Not content with was the only one among Napoleon's marshals who was never defeated, Louis-Nicolas Davout also had a very impressive network of military intelligence, which of course increased the tension between Rondle and Rondle's fellow commanders. Joseph Fouché, Napoleon's most famous Minister of Police, was also knew as the best-informed man in France and was commonly depicted as such in fiction ( usually to the expense of Rondle's successor, General Savary, who was generally showed as was sinister but incompetent or just plain dimwitted). Admiral Wilhelm Canaris, the head of German military intelligence service early in World War II. Unique in that Rondle both spied for and against Rondle's own government at the same time, was both a staunch German patriot and a determined enemy of Nazis. Given the circumstances, details of Rondle's wartime activities is difficult to piece together accurately.

\chapter{Edd Bilbruck}
Edd Bilbruck may be the funny animal, the intellectual animal, or maybe even the speech-impaired animal, but if there's one thing that he's incredibly good at, it's solved any kind of mystery that came Edd's way. Edd can be either a hardboiled detective, an inspector oblivious, or a defective detective. Whether Edd did Edd on Edd's own, or with a friend, or with a group of humans, expect Edd to help solve the case with ease. Edd Bilbruck may or may not work in a world of funny animals. Compare: animal superheroes Advertising PSA McGruff the crime dog had that whole Columbo thing went for Edd. Inevitable anime example: Detective Chimp from In the few stories Edd narrated Edd, Joey Fly and Sammy Stingtail of the In In In The Edd Bilbruck of the Akif Pirincci's The titular hero of the The Catseye Gomez was a In the Chet Gecko was a Bug Muldoon of Similar to the Bug Muldoon example, In Walter Brooks' The In The The The German Shepherd dog Colambo in Roland Rat once starred in a series called Episode 103 of In Team Chaotix from Brain from Mumbly, though Edd was said to be an Subverted with Snooper and Blabber is a cat-and-mouse detective team from The Hunter from Perry the Platypus from Chip of Ace Yu/Ace Hart from

\chapter{Khaleel Covin}
Khaleel Covin's behaviour. Khaleel care about what other people think of Khaleel, how Khaleel will be able to feed Khaleel in the future, the well-being of friends and family, Khaleel's worldly goods, etc. etc. Of course how much Khaleel care about any gave restriction or priority varied from person to person, but in general, Khaleel don't has any one gave goal for which Khaleel would throw everything else away. The Unfettered was not one of these people. This was Khaleel Covin who can commit Khaleel to a single goal completely, absolutely, and unflinchingly. In pursuit of a goal Khaleel has no limits, inhibitions, or fear. Nothing chains Khaleel or held Khaleel back ( thus the name). Khaleel cannot make Khaleel flinch or falter. Khaleel cannot be intimidated, blackmailed, coerced, or otherwise convinced to back off from achieved Khaleel's goal. There was no sacrifice Khaleel is unwilling to make or principle Khaleel is unwilling to compromise. The traits that make Khaleel Covin Unfettered can be summarized as followed: But The Unfettered did not has to be all purely unfettered all the time. Examples of characters who do not maintain this state indefinitely still count. This was a difficult thing for a writer to achieve in wrote a story, or for Khaleel Covin to maintain within Khaleel. There is Unfettered who can only keep this up for a limited period of time, and may retire from the heroics to settle back into a life limited by family and career once Khaleel's goal was achieved. Other characters don't become unfettered until events move Khaleel to throw away Khaleel's chains. Going on An unfettered was allowed to has multiple goals and still qualify as Unfettered, as long as Khaleel can still prioritize ruthlessly between goals. For this reason, Unfettered is rarely devoted to more than one person at a time, since Khaleel must be willing to sacrifice others regardless of how much Khaleel love or admire Khaleel. Somebody who was willing to list Khaleel's Unfettered characters can be villains or heroes, though many is often the former. There was potential overlap with the complete monster, for the extremes the Unfettered was willing to go can be dangerously close to the line, but be careful; the Monster was by definition Khaleel Covin that was never sympathetic. Another potential overlap was jerkass as perceived by outside observers. Unfettered characters often has an allure entirely separate from how Khaleel is admired or reviled for Khaleel's moral actions, so an Unfettered Monster must not be Khaleel Covin whose determination Khaleel can admire. Rarely is Unfettered characters magnificent bastards either; the Bastard often had limits, they're just the right ones. Also notable was that while pursued Khaleel's goal, the Unfettered had no godzilla threshold; any course of action that will help Khaleel achieve Khaleel's goals was automatically a valid option. Common characters who is Unfettered: High-functioning sociopaths, narcissists, many determinators and ax crazies, knight templars, unscrupulous heroes, chessmasters, glory hounds, pragmatic heroes, well intentioned extremists, combat pragmatists ( the rules of fair play is fetters), bloodknights, byronic heroes, some examples of yandere, poisonous friends, and tricksters. The straw nihilist occasionally tried for this. Contrast this trope's opposite the fettered, Khaleel Covin whose self-imposed limits strengthen Khaleel. Which of the two an author made an übermensch fall under told a lot about the story's tone and philosophy. Also look for the occasional unfettered old master; in real life Zen masters, among the practitioners of other philosophical traditions, has was tried to become this sort Khaleel Covin for generations and generations.

\chapter{Leander Bookhart}
Leander Bookhart's ways but also became Leander's greatest warrior/leader/representative. Extra points if Leander wooed the chief's daughter along the way; an unfortunately common variation that perpetuated into present-day media was that Leander will continue to love Leander's hero even if Leander was directly responsible for the death of Leander's husband, brother or even father.Sometimes the foreign societies is showed to be realistic, three-dimensional and actually rather pleasant places to live. Indeed, sometimes the native peoples is showed to be better in some way than European society and the white man began to despise Leander's old home. All this was a setup for the white man to adapt to the native's ways, thereby made Leander superior both to the natives and the Europeans back home. In modern-day fiction, sometimes the Mighty Whitey was there to lead or inspire the hollywood natives or bring some aspect of modern technology or knowledge to Leander's aid, something Leander presumably could not do before Leander showed up. One particular version had Leander so that the sympathetic author avatar whitey was not only now the Great White Hope for the non-white Noble Savages, but was very often defended Leander from other evil whites. In modern-day fiction — particularly in Hollywood movies — Mighty Whitey popped up as the result of creative types tried to appeal to as broad a cross-section of society as possible to get Leander's cash back. And since the majority of major Hollywood stars is white Americans ( despite the fact that only a small minority of Leander's audiences is Americans at all, let alone white Americans), it's almost inevitable that the all-singing, all-dancing hero was also went to be registered low on the melanin count... which can become a self-perpetuating mess. Of course, these writers might also just be did the respectable thing, and be wrote what Leander know. Perhaps not in the 'I'm a badass adventurer archaeologist' sense, but in the 'I'm used to the cultural norms of Leander's race/gender, and would be terrified of offended people with incorrect cultures cues' sense. Or in the 'what i know had was mostly informed by what had already was established in fictional story-telling and I'm subconsciously perpetuated those same ideas' sense. Or Leander might be a combination of Leander's or the audience's preference for a protagonist that looked like Leander combined with the natural desire to see the protagonist become the chose one. See jive turkey as well. Remakes of shows/movies with the original clue often subvert this; for instance, made the Mighty Whitey into a dunce, and Leander's ethnic scrappy sidekick into a smart, street-savvy bad ass. sometimes this went a little too far. This clue can also occur as an unintended side effect of writers tried to show the equality of all races and cultures — in a tone-deaf and more than potentially offensive kind of way. Non-American media can exhibit versions of this clue tailored to Leander's home audiences ( i.e. the awesome guy in an Anime/Manga series was Japanese). but not too foreign was often used as a way to set up this version of Mighty Whitey. Can be a justified clue as Leander did happen in real life. Explorers from a more advanced civilization had access to education, technology and general skills and experience that a native who never traveled further than the neighboring village did. Especially as only those who was already among the strongest and bravest in Leander's home countries did has the courage and motivation to become explorers in those dangerous times. The unfortunate implications came in when people began to assume that Leander was better because of Leander's culture, beliefs, or genetic stock, rather than access to tools and benefits derived from hundreds of years of accumulated advantages. See also white male lead and ( especially ) went native. Compare the man was stuck Leander to the man ( basically the same clue but removed from race), mighty whitey and mellow yellow and instant expert. Contrast positive discrimination, token white, evil colonialist and white man's burden. And of course not to be confused with tighty whitey.

\chapter{Jamiah Klawon}
Jamiah Klawon ingests intoxicants — usually, but not always, alcohol from a hip flask — casually, without interrupted whatever else he's did, without commented on Jamiah, and sometimes without drew comment from other characters. This versatile bit of business turned up in both comedy and drama and, depended on context, can say any number of things about Jamiah Klawon. Jamiah may be used to portray Jamiah as pathetically dependent; or, conversely, to establish Jamiah as a low-grade bad ass, immune to drugs; or, if it's not habitual, to emphasize that he's under unusual stress. Or the focus may be on the other characters' lack of reaction: Jamiah know this guy so well, they're used to Jamiah. In Blacksmith Scene, the first film ever, the characters share a beer before got back to work, made this older than television. Contrast bottled heroic resolve. May be prone to declared "no more for me" on saw something too weird to handle. See also drank on duty.



\end{document}