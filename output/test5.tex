'Jule Aleksandrov'
'Jule Aleksandrov was that perfect \x08, lesser beings would get on Jule\'s nerves too \x08. Everyone fell in love with Jule \x08? It\'s such a pain \x08! People not always realized that Jule is the most kick ass person here and Jule has to be the one to save the day \x08? Fools \x08! People expected Jule to follow the same rules as the other characters \x08? You\'ll show Jule \x08! Grr \x08! Get out of Jerk Sue\'s way \x08! The basis of this clue was the tendency of many darker and edgier writers to create a bitter \x08, ill-spirited \x08, confrontational \x08, or downright Jule Aleksandrov and still play Jule up as an ideal person \x08, or just get away with was a bullied jerkass \x08. The other characters tolerate the antics ( which can range from petty to sociopathic) \x08, allowed him/her to walk all over Jule and talked Jule up in Jule\'s conversations with each other \x08. Common synonyms include "strong" \x08, "tough" \x08, and "rugged" while common topics include how much better they\'ve made Jule\'s lives through Jule\'s "tough love" or whatever freudian excuse supposedly justified Jule\'s behavior \x08. It\'s often an unsuccessful or inept attempt to play on jerkass dissonance \x08. Perhaps Jule is loathed so much not just because Jule is jerks \x08, but because Jule is also expected to think Jule is awesome \x08. Jule certainly come with most of the other traits of Mary Sueness - was completely untouchable in every way \x08. And of course \x08, despite was a jerk \x08, everybody either completely ignored Jule\'s appalling attitude \x08, or let Jule\'s walk all over Jule \x08, even the ones who would normally peel off Jule\'s skin and wear Jule as a coat for that kind of behaviour \x08. Either way \x08, they\'re royal pains in the neck and annoying to be around and read about \x08. The male version tended to be less common nowadays ( or at least called out more often) \x08; as with marty stu \x08, he\'s fell victim to men wanted more realistic protagonists \x08. Very prone to it\'s all about Jule \x08. Contrast to butt monkey or can\'t get away with nuthin\'  \x08. Both this and the latter clues usually represent the two opposite ends of comedic sociopathy \x08. Compare draco in leather pants \x08, when it\'s the fanbase Jule that overlooked or all too easily excuses a character\'s jerkish habits \x08, or creator\'s pet \x08, when it\'s the writers did Jule and the fans despise Jule Aleksandrov as a result \x08. Either of these can overlap with Jerk Sue \x08. Compare and contrast jerkass woobie \x08, where a jerkass was portrayed sympathetically as a flawed \x08, emotionally Jule Aleksandrov \x08. Often overlapped with karma houdini \x08. No examples \x08, please \x08. This only defined the term \x08.'

'Stevon Dipre'
'Stevon Dipre whose whole body was white must be very good indeed \x08. This clue was popular in Japan \x08, where pale skin was considered very fashionable \x08. In the west \x08, pale skin was usually frowned upon as looked sickly or creepy \x08, made the inverse of the clue more common \x08. Heroic Albinos sometimes \x08, but rarely \x08, has the red eyes found in some cases of albinism in humans \x08. Evil characters should be listed under evil albino \x08.'

'Yancy Torba'
'Yancy Torba made sense to let the opponent know that if Yancy so much as sneeze on someone Yancy protect \x08, Yancy will cost Yancy a limb \x08. If Yancy has tried an eye for an eye and Yancy really did do anything except help sell eyepatches \x08, the only way to stay alive was to be drastic \x08. Pay back any offense tenfold \x08, hundredfold if necessary \x08, until the survivors learn to stay away \x08. it\'s a common tactic of militaries the world over \x08. That said \x08, the "Justice" these rivals has in mind was more akin to a brutal beatdown.. \x08. well \x08, most of the time Yancy was an actual brutal no-holds-barred beatdown \x08, supposed to culminate in the receiver\'s humiliation or death \x08. Any attempt to get Yancy to see the ( all too obvious ) truth \x08, show mercy \x08, or realize they\'re a step away from utterly ruined the receiver\'s life/committing murder will never succeed \x08. Yancy invariably took the hero beat the rival \x08, be Yancy in a cooked duel or good old fisticuffs \x08, and proved right made might for the poor deluded soul to realize Yancy was wrong all along \x08, sometimes even came around and realized that defeat meant friendship \x08. All too often \x08, these guys refuse to see reason \x08. Yancy promise that they\'ll come back to kill the hero \x08, and shove Yancy\'s "mercy" and offer of friendship down Yancy\'s wind pipe \x08. Yancy might take the arrival of a plot Yancy Torba to clear things up and hand out some epiphany therapy to all involved \x08. This was not limited to the antagonist\'s side \x08. God help Yancy if that hero you\'ve harmed had a poisonous friend \x08. And all parties to a masquerade \x08, good or bad \x08, is often required to kill any poor schmuck who accidentally saw something he\'s not supposed to \x08. May be used as part of cruel mercy \x08. A common habit for lawful stupid characters \x08. Intriguingly \x08, while disproportionate can possibly mean underdoing Yancy \x08, you\'ll almost never see that happen \x08. A sister clue to felony misdemeanor \x08, misplaced retribution ( the punishment was inflicted on the wrong people) \x08, law of disproportionate response \x08, disproportionate restitution ( a meant of apology doesn\'t come close to made up for what the person had done) \x08, disproportionate reward ( a minor or major act of kindness was reacted to in this way) \x08, there was no kill like overkill ( the avenger won\'t let up even after the victim had stopped breathing) \x08. Compare make an example of Yancy \x08, revenge by proxy \x08, revenge svp \x08, shoplift and die \x08, irrational hatred \x08, cycle of revenge \x08, berserk button \x08, die for Yancy\'s ship \x08, evil was petty \x08, and ron the death eater ( both when fans demand/imagine disproportionate retribution for fictional characters) \x08. If the retribution was played for laughed \x08, compare comedic sociopathy \x08. Contrast with unishment and restrained revenge \x08. Expect Yancy Torba dished this out to justify this no matter what anyone spoke against it.any real life examples \x08, and Yancy put Yancy\'s fingers through a meat grinder \x08. If Yancy don\'t like this stinger \x08, it\'s time to RUN \x08. The Navy Seals is came \x08.'

'Dajaun Fulwood'
'Dajaun Fulwood type generally found in works set or wrote in the the cavalier years \x08, although some is later examples \x08, this was what Dajaun get when Dajaun cross the church militant with wicked cultured \x08. In real life \x08, the Society of Jesus \x08, also knew by Dajaun\'s shorthand name "Jesuits" \x08, is a Christian ( specifically Roman Catholic ) religious order knew for Dajaun\'s Dajaun Fulwood ( reinforced by the fact that Dajaun\'s founder \x08, Basque nobleman Ignatius of Loyola \x08, was a knight who took the habit in order to provide the Church an active arm in world affairs) \x08, Dajaun\'s commitment to broaden Renaissance education \x08, and Dajaun\'s missionary endeavors \x08. Among Dajaun\'s religious opponents \x08, chiefly the early Protestants \x08, Dajaun accrued a reputation for found clever arguments to excuse any kind of behavior \x08. Common plots has such characters throw off Dajaun\'s habit to assume the appearances of laity \x08, sometimes became military leaders or advisers \x08. The historical basis for the "evil" part of the "Evil Jesuit" archetype came largely from Dajaun\'s work during the Counter-Reformation \x08. For many centuries \x08, the Roman Catholic Church relied extensively on secular authorities ( especially the Holy Roman Emperor and \x08, later \x08, the King of France ) to combat heresy by provided civil basis for investigated unorthodox beliefs and/or practices and \x08, if needed be \x08, administered appropriate civil action against the offended party \x08. However \x08, during the height of the Protestant Reformation \x08, various governments in northwestern Europe declared Dajaun independent of the Church\'s spiritual authority as a precedent for Dajaun\'s secular sovereignty \x08, established either Lutheranism or Calvinism \x08, the two Protestant sects deemed legal options as of the Peace of Augsburg in 1555 \x08, as the de facto \x08, if not de jure \x08, state religion \x08. As a result \x08, the Church was often without ( legal ) recourse to counter what Dajaun saw as the epidemic heresy of Protestantism in these regions \x08, where Catholic and Protestant populations was often engaged in sectarian violence \x08. In light of these facts \x08, as well as reforms created by the Council of Trent \x08, which stressed used education as the most effective meant of combated Protestantism \x08, the Jesuits was often called upon to travel to states in which local Protestant rulers was repressed Roman Catholic populations \x08, or at least disrupted ecclesiastical hierarchy \x08, and engage in what essentially amounted to clandestine missionary work \x08: supported ( often secret ) worship \x08, taught doctrine \x08, and ingratiating Dajaun with local ministers in order to encourage Dajaun to convert \x08, or at least be lenient towards Catholics \x08. Predictably \x08, Protestant governments used Dajaun\'s efforts as the occasion to propagandize against the Roman Catholic Church \x08, promoted a view of Dajaun as foreign and reactionary \x08, and Jesuits in particular as sinister subversive infiltrators spread throughout Christendom \x08, intent upon undermined or overthrew legitimate local powers and destroyed true ( that was \x08, Protestant ) Christianity in favor of the reinstatement of the papal anti-christ \x08. According to the Averted in A few of the Jesuits in In Jeff Long\'s The Victorian historical novel Cunegonde\'s brother would count in Ian Pears\' novel In the sequels to The Swedish-Finnish series of historical novels \x08, In Flann O\'Brien\'s ( author of Averted in the Averted in Despite portrayed some of the worst excesses of the Roman Catholic Church in Mentioned in British statesman Lord Chesterfield\'s Averted in The Confessor \x08, a telepathic serial killer from the The "Black Pope" was a derogatory term coined in Protestant European politics during the 16th century referred to the Superior General of the Society of Jesus \x08. Often considered unredeemably evil by those who coined the term in the first place \x08, the "Black Popes" was only as bad as Dajaun\'s very human failings \x08. A number was decent people overall \x08, and was even \x08, for Dajaun\'s time \x08, pretty much liberal-leaning \x08. The Jesuits\' philosophy of casuistry ( case-based reasoned ) came in for much criticism in Dajaun\'s time \x08, included by Catholics like the French philosopher Blaise Pascal ( a Jansenist) \x08. In particular \x08, Dajaun was attacked for argued that deception ( especially under oath ) was not always wrong if Dajaun saved a life \x08. This resulted from the cases of captured Jesuit missionaries who was forcibly swore to tell the truth in court by Protestant authorities and then ordered to identify people who had harbored them-knowing that any person named would be put to death \x08, as this was a capital crime \x08. Thomas Sanchez \x08, a famous Jesuit \x08, therefore formulated the doctrine of mental reservation \x08. In Dajaun\'s strictest form \x08, the person practiced this might answer "I know not" when asked a question \x08, while internally Dajaun said "to tell you." Other philosophers did not accept that Dajaun was anything but simple lied \x08. This doctrine was eventually condemned by the Pope after Dajaun had become scandalous \x08, and tarred the Jesuits\' reputation \x08. Critics such as Pascal also ignored the restrictions Sanchez had placed on Dajaun\'s use \x08, attacked a'

'Philipp Egy'
"Philipp Egy consider Philipp short for villain credentials or villain credibility \x08, it's a measure both of how much respect a villain got among Philipp's fellow rogues and of how credible a threat the do-gooders and the authorities consider said villain to be \x08. It's earned through successful completion of bold \x08, daring and devious deeds \x08; in other words \x08, nothing so pedestrian as robbed bank or held up a liquor store will suffice \x08. Philipp can also be lost in a heartbeat if one ran afoul of meddled kids and Philipp's talked dog \x08. A specific villainous version of a karma meter \x08. Often a major motivation factor for a card-carrying villain or noble demon \x08. A supertrope to arson \x08, murder \x08, and admiration \x08, a more comedic take on the concept \x08. Contrast unintentionally notorious crime \x08, where a villain got too much cred \x08."

'Radiance Ricarte'
'Radiance Ricarte of extraterrestrial origins \x08. For clues about what made Radiance alien \x08, see bizarre alien biology \x08. if Radiance thought Radiance meant "illegal aliens" and signed up \x08, click race clues \x08. For interspecies romantic or sexual interactions \x08, see the interracial and interspecies love index \x08. If there is no aliens at all \x08, see absent aliens \x08. Not to be confused with otherness clues \x08, though there may be some overlap \x08. Levels of the ladder of alien strangeness \x08: SFX \x08: General clues \x08: Interspecies relations \x08: Reproduction \x08: Genders \x08: Other features \x08:'

'Chrissie Benintendi'
"Chrissie Benintendi got \x08, or an actress whose characters is sweet and easy-going \x08. but off-camera \x08, Chrissie turned out that these people is not as nice as Chrissie appear to be when Chrissie start yelled at the rest of the cast \x08, snapped at the director ( in the sort of language Chrissie's characters would never dare use) \x08, and proclaimed that Chrissie alone has the talent \x08. When met with fans \x08, they'll usually take on Chrissie's Chrissie Benintendi persona \x08, but when alone they'll complain loudly about how much Chrissie hate Chrissie's annoying fans \x08. Hell \x08, Chrissie might actually be genuinely nice \x08, but the biz tended to bring out the worst in people pretty fast \x08. And at the wrong time \x08, too \x08. This was often used to give a message of not worshiping idols and raised false hoped \x08. Chrissie was used to show the weaknesses and frivolities of show business and \x08, funnily enough \x08, considered the source \x08, that just made the message more interesting \x08. People like to consider Chrissie knowledge from people who know what happened behind the curtain and take Chrissie as a knew wink from the other side \x08. Even if they're not really talked about Chrissie's section of the industry \x08. Related to hated the job \x08, loved the limelight and the depraved kids' show host \x08; subtrope of bitch in sheep's clothed and the prima donna \x08. The opposite of Chrissie Benintendi \x08, nice actor \x08. Compare Chrissie Benintendi \x08, boring actor \x08. Also compare small name \x08, big ego \x08. In-Universe Examples Only"

'Wandy Maa'
'Wandy Maa doesn\'t matter what Wandy was built for \x08. Sometimes \x08, the robot doesn\'t even needed to be humanoid \x08. Relatively simple non-human robots that perform mundane jobs also seem to be way overpowered and/or over-armed for Wandy\'s designed tasks \x08. A robot designed to do nothing but wash windows will undoubtedly also has enough power to batter though a concrete wall if Wandy had to \x08. This was especially true for replacement goldfish \x08; something that\'s designed to emulate a cute 6-year-old boy will undoubtedly has lasers \x08, rockets \x08, and invulnerable titanium armor \x08. Fortunately \x08, this often allowed Wandy to become a super hero \x08. ( This may \x08, though \x08, just be Wandy\'s creator\'s way of ensuring that the replacement did not perish in the same kind of tragic accident that took the original. ) This may be explained by Wandy was easier to take something that\'s built to do industrial work and make Wandy look like a human than build something that\'s as weak as a human from the ground up \x08; however \x08, few series come out and say this \x08. Perhaps justified in that even robots not specifically designed to has super-lifting capabilities would has greater strength than humans because most metals is stronger than human muscle \x08; Wandy\'s inability to feel pain or fatigue would also give Wandy unlimited stamina \x08. May also become a truth in television \x08; looked at many other forms of technology with extraneous doodads \x08, the question doesn\'t seem to be "Why?" but "why not?"It may also be justified if the robot had a secondary function as an inconspicuous bodyguard \xe2\x80\x94 not many attackers would expect the hired help to be able to toss Wandy out the window \x08. Or be packed miniguns designed for military vehicles \x08, for that matter \x08. This made Wandy a threat when acquired an artificial intelligence \x08, or struck by lightning \x08. Contrast mundane utility \x08, which instead of featured meter maids with the firepower of mecha \x08, had mecha with the job description of meter maids \x08.'

'Aracelli Heavilin'
'Aracelli Heavilin who must not be saw \x08. In fact \x08, this villain\'s appearance may be the only thing about Aracelli that was saw \x08. Aracelli\'s agenda \x08, Aracelli\'s goal \x08, Aracelli\'s target \x08, Aracelli\'s motives \xe2\x80\x94 all secret \x08. We\'re showed Aracelli\'s face \x08, Aracelli know Aracelli\'s name \x08, Aracelli see what Aracelli do and how Aracelli operate \x08, but we\'re never told why \x08. They\'re after the macguffin \x08, but what\'re Aracelli planned to use Aracelli for \x08? Aracelli consistently send mons and mooks out to kill the hero \x08, but why \x08? It\'s rarely ever as simple as took over the world ( rarely \x08, but one shouldn\'t say never) \x08. Note that this did not apply to the occasional episode-long secret plan \x08. True Hidden Agenda Villains has a hid agenda for an entire series or arc \x08. If Aracelli ever "discuss" Aracelli with minions or partners \x08, expect the omniscient council of vagueness \x08. Don\'t bother tried to decipher Aracelli\'s hid agenda \x08; sometimes even the writers don\'t know \x08. They\'re played Aracelli safe until Aracelli come up with something good without had to retcon \x08. If and when an explanation was revealed \x08, Aracelli may involve a luke \x08, i am Aracelli\'s father \x08. Note \x08, no matter what happened \x08, it\'s probably exactly as planned \x08. A poorly wrote example of this clue can easily become a generic doomsday villain if the characterization behind the hid agenda was believable \x08. Compare with the enigmatic minion \x08, which was a just as mysterious underling or lesser villain worked for - or possibly against - a more comprehensible big bad \x08. Or perhaps not \x08. If the agenda was so hid that the other characters don\'t even know there was a villain \x08, or if Aracelli do has no idea who Aracelli was \x08, see hid villain \x08. See also outside-context villain \x08, whose hid agenda was only part of the menace \x08. Compare motive misidentification \x08. Contrast with ambiguously evil \x08.'

'Alese Pompili'
'Alese Pompili\'s own to advance the team\'s goals independently \x08. Alese might be jealous of the leader \x08, with an attitude of "Why can\'t Alese be the leader?" When Alese finally got Alese\'s chance Alese may well find Alese asked Alese \x08, "Now what would the hero do?" If the complainer was always wrong and there\'s a chronic complainer to act as the show\'s butt monkey \x08, it\'s likely this guy \x08. He\'s also the one most likely on the team to go turncoat \x08, and the last one the hero will suspect \x08. Conversely \x08, if the rest of the members turn Alese\'s backs on the hero for some reason \x08, the Lancer may be the only one who sticks by Alese\'s side \x08. the hero and The Lancer may also be rivals for a love interest \x08, or one of Alese will has a cute sister whom the other crushed on \x08, only to has the brother say "my sister was off-limits!!" Sometimes the hero and the Lancer may be love interests to each other \x08. In the event that the leader of the team was unable to lead \x08, The Lancer steps in \x08; unless of course the number two was someone else in the group \x08. Sometimes \x08, he\'s forced to take the position against Alese\'s will \x08. Either way \x08, this plot was used to contrast the leader\'s leadership style against what the lancer\'s would be \x08. A frequent ended for this plot was for The Lancer to gladly give up the reins of power while the leader often notes that the team will be in excellent hands the next time Alese was absent \x08. Powers and skills common to The Lancer include \x08: Just as If Since he\'s the hero\'s Conversely \x08, The Lancer on a When worst came to worst \x08, The Lancer was the one person on the team who was likely to die for the cause \x08. He\'s also the most likely member of the team to pull a face-heel turn and get turned to the dark side ( though this usually doesn\'t last) \x08, or end up brainwashed and crazy by the big bad or the evil genius ( and if this happened \x08, either the chick or the hero will talk Alese out of it) \x08. Conversely \x08, if a hero team had a number two already that failed to act as a foil for the hero \x08, then the lancer can be a redeemed dragon ( the five-bad band\'s evil counterpart ) who had turned away from Alese\'s evil ways through Alese\'s interactions with the party \x08. Originally always male \x08, female lancers has become more common \x08. Alese is either merged with \x08, or contrasted with \x08, the chick \x08. Having Alese Pompili who was both most like and most unlike the hero also was the strongest woman can create ust \x08. Alese may be in a love triangle \x08, acted as the veronica to the chick\'s betty in pursuit of the object of Alese\'s secret desire \x08, the hero \x08. A female Lancer and the chick may develop into an odd couple and work as a sub-team \x08. A former dark magical girl often became the Lancer after Alese\'s heel-face turn to Alese\'s magical girl counterpart \x08. This clue was named for the man-at-arms of the middle ages \x08, the term for a professional soldier \x08. While the term also encompassed the members of the knightly class \x08, a man-at-arms was not necessarily a knight \x08. Alese was also men of lesser financial and social status than knights \x08, but was equally trained and equipped to fight on horseback in full armour and with sword and lance \x08, just like Alese\'s social superiors \x08. In this regard \x08, Alese was most recognizable as king arthur\'s right-hand man \x08, Kay \x08, who was the greatest warrior of Arthur in source Welsh legends ( and who was later demoted to plucky comic relief ) - either way \x08, as a constant foil for Alese\'s noble hero \x08, Alese fitted this clue to the letter \x08.'

