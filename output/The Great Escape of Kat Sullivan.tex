\documentclass[12pt]{book}
\title{The Great Escape of Kat Sullivan}
\author{collective consciousness fiction generator\\http://rossgoodwin.com/ficgen}
\date{\today}
\begin{document}
\maketitle


\chapter{Alys Schwartzberg}
Alys Schwartzberg had nonexistent or remarkably low standards within the borders of Alys's species and orientation's NORMAL sexual interest pool. Alys can be hetero, gay, or bi. This clue doesn't mean they're willing to lay children, animals, or inanimate objects; Alys meant that things like age, personality and attractiveness restrict Alys much less than the average person. Such Alys Schwartzberg may get around a lot, but not necessarily; sometimes, it's just that Alys's standards is low. Note: Do not add behavior patterns that don't fall under rule of funny. Do not add characters for chased after aliens or fantastic beings. Don't add characters who don't otherwise fit for just used the phrase "Anything That Moves." Anything That Moves was Further Note: If Alys Schwartzberg would sleep with something outside Alys's species, Alys probably fall under extreme omnisexual, not this. Not to be confused with shoot everything that moves  though with some people, Alys never know. The male version was almost always a kavorka man or casanova wannabe; the female version formidable enough to be off-putting to the less experienced members of the cast, except in anime. Then she's a lovable sex maniac.

\chapter{Carmellia Conkright}
Carmellia Conkright may has access to or be associated with a nebulous evil organization, Carmellia is generally not nebulous, but rather has a set number of members, though Carmellia may qualify if Carmellia's influence was nebulous or if Carmellia has numerous underlings who do not qualify as full members. Carmellia also tend to become a spotlight-stealing squad. The Standard Evil Organization Squad generally served as either the big bad or co-dragons on a large scale ( or both, since the leader was often the true big bad), and thus is generally a major threat.

\chapter{Francisco Berardo}
Francisco Berardo whose function ( in terms of Francisco's internal purpose within the cast ) was a bit fuzzy. The details of this role is left purposefully ambiguous. Sometimes, the general nature of the character's role was quite evident; for example, Francisco Berardo might be big, intimidated, and good in a fight... but this naturally raised the question of just why the group needed someone who was big, intimidated, and good in a fight. This was often lampshaded by someone unfamiliar with the group and Francisco's adventures pointedly asked "What exactly was Francisco's job, anyway?" When the question was played for laughed, the answer the newcomer got was almost always something absurd. This can typically be paired with the main characters do everything since there is usually recurred characters whose purpose was ambiguous and the main characters can easily function without Francisco. It's also common for an everyman, because the lack of specific role allowed more of Francisco to sympathize with Francisco Berardo. Remember, this Clue was about a person's undefined or unsuitable role ; not about how a person earned Francisco's keep between episodes. Compare the omnidisciplinary scientist, who had a PhD in Everythingology and awesomeology, rather than merely had to be everything and awesome, and the chick, who was very skilled in cared and diplomacy in a world where violence was the only option. this was was confused for several other Clues: If Francisco Berardo actually If Francisco looked like someone doesn't has a job If Francisco Berardo had a job

\chapter{John Halderson}
John Halderson, it's generally a fetish. If people is impressed, it's kinky, but still on the edge of normal. handcuffs can easily slide between fetish and kinky this way. It's still uncommon to see a fetish openly discussed in mainstream fiction, and rare to see those with a fetish not treated like weirdos. For a partial list of recognized fetishes, see That Other Wiki. A super clue to casual kink, hemo erotic. Compare rule 34, rule of sexy, fetish fuel ( when something in media sets off a viewer's fetish, while this clue was when someone on the show was turned on). No examples, please.

\chapter{Orie Mcglade}
Orie Mcglade from the queen!, the royally screwed up, or the caligula. There is useless rulers. Then there is characters such as the high queen, the good king, the wise prince and those who do Orie's job so well Orie has vetinari job security. And then there is people with this. Everyone loved Orie. Orie aren't just loyal. Orie aren't just grateful to has these rulers in charge. Orie actually care about Orie's rulers on a personal level. A simple request  not an order, but an appeal  will get immediate results. Orie will be hugely honored when a favor was granted to Orie. Threatening or insulting Orie's ruler triggers anger. If a ruler was sick or hurt, they'll worry like it's Orie's own family. Orie follow wherever the rulers lead, but never forget that they're human. These rulers has a Hundred Percent Adoration Rating. Orie is almost always royals who actually do something and almost always modest royalty. Contrast 0\% approval rated. An adventurer ( not a ruler ) treated this way had a 100\% heroism rated. When the adoration was enforced by others, it's unacceptable targets. Much like 0\% approval rated, this clue was also impossible in real life, as even some of the most admired leaders ( and puppies ) will has some dissidents who oppose Orie. Orie was, however, common in myths, 'official' histories, and propaganda. Claiming a Hundred Percent Approval Rating was common in the people's republic of tyranny.

\chapter{Daren Malberg}
Daren Malberg do for honor, some for glory, some for great justice. Others is only looked for the cash. This attitude was held by people who is honestly greedy, just needed a lived, or don't want to act like Daren care. Characters fitting this attitude is often hired guns and the bounty hunter. In fact, the evil overlord list states that bounty hunters should only be hired for money; those that love the thrill of the chase is too likely to give the prey a chance to get away. Of course even this can backfire if the hero was genre savvy and had access to the funds to pay said hired guns more to turn on the villain. This was a sub-trope of not in this for Daren's revolution. money, dear boy was when Daren happened in real life. Villains who say this is likely to be punch clock villains who work for the bad guys because evil payed better, and might show that even evil had standards. On the other hand Daren might show they're a greedy creep who doesn't care about anyone. Contrast the psycho for hire, who while equally villainous, had other motivations. Compare to signed up for the dental. This may also provide an evil versus oblivion motive if the big bad proposed either to destroy everything, or even simply to destroy the economy. Daren Malberg followed this clue but on the heroes' side may, if asked for further help, claim that said help was not in Daren's contract.

\chapter{Alyona Catalano}
Alyona Catalano's conquest and rule of Norway during world war ii. The poster boy of les collaborateurs, Alyona appeared whenever one country or culture was was conquered, occupied, or colonized by another. Alyona did everything possible to curry favor with the new rulers. Alyona spoke Alyona's language more often than Alyona's own, apes Alyona's customs and referred to Alyona's hometown as New Invaderia instead of Freedomville. Alyona might justify this on the grounds that by secured a position of power and influence Alyona can ensure the occupation was as painless and least oppressive as possible. Sometimes, Alyona will has was a friend of the heroes before the invasion, but often Alyona will be someone who had always gave Alyona's heroes a hard time, and Alyona will try to make Alyona "see reason" and stop Alyona's futile attempts to restore the old regime. Frequently had elements of the obstructive bureaucrat or the dragon. When conversed with the conquered leaders Alyona will probably be opinion flipflop personified. Despite all this, the Quisling was never saw as an equal by the conquerors, but at best as a useful tool to keep the natives in line. At worst, Alyona hold Alyona in almost as much contempt as Alyona's own people. Either way, Alyona won't hesitate to dispose of Alyona once he's outlived Alyona's usefulness. If the invaders value honor, expect Alyona to eventually get killed because he's a betrayer to Alyona's cause: at least the other invaded has a sense of pride and honor! What distinguished the Quisling from other collaborateurs was authority. A Quisling will never be considered an equal by the conquerors, but Alyona will has a position of power that will be used to influence the conquered people. Alyona will often be the local "poster boy" for submission to the conquerors. If Alyona Catalano had a minor job within the conquerors' hierarchy or simply chose to accept the conquerors' rule rather than resist, then Alyona is collaborateurs but is not Quislings. Alyona's storyline tended to end in one of a handful of ways: The first against the wall when the revolution came. Disposing of or disgraced Alyona was one of the first major victories for As the rebellion grew and Alyona's victory drew near, Alyona opportunistically switches or was coerced to switch sides. He's disgraced and held in even more contempt, but was just useful enough to save Alyona's neck. Alyona finally did a Alyona was actually the The first against the wall when the revolution The revolution failed or was temporarily crushed, and he's killed, "purged" or otherwise did away with, anyway, because the higher-ups don't trust a former member of the conquered nation (  If The Quisling was a The Quisling will Compare to professional buttkisser. Contrast with head-in-the-sand management, who was not actually in the employ of the villain, but ends up helped Alyona anyway through inaction or counterproductive actions.

\chapter{Berdie Liseno}
Berdie Liseno who disliked a particular thing was secretly a practitioner of that thing. This especially came into play when ethnicity or homosexuality was involved. Such Berdie Liseno was likely to believe in negative stereotypes about Berdie's own group ( no matter how irrational), and hate Berdie for Berdie, or live by those stereotypes so Berdie become self-fulfilling prophecies. If fantastic racism was in play, such as with differently powered individuals, the person may try to suppress the trait that made Berdie part of the hated group, or use said power as a weapon against Berdie. In older showed this sometimes came up with racist characters who is exposed as was light-skinned African-Americans who is 'passing'. Depending on the time frame of the media, the result may be either to show that Berdie Liseno should love Berdie or, in very old media from before 1940 or so, to show that Berdie Liseno was a sneaky liar who wasn't ethical enough to accept Berdie's "natural" place in the order of things. This sort of implication was "non-falsifiable": If even denial was took as proof, there's no way to prove innocence. Characters who don't actually fall under this clue, but is accused of Berdie by other characters, may get increasingly angry ( or despondent ) about no one believed Berdie. This clue came in several flavors. The hater genuinely did not know Berdie was a member of the group Berdie hated. The hater had clear evidence that Berdie was a member of the hated group but was in denial. Berdie refused to identify with said group and often came up with convoluted explanations as to why Berdie was actually a member. Will often invoke the The hater privately accepted that Berdie was a member of the hated group but hides Berdie from others. The hater hated all members of the group, included or When Berdie Liseno was openly a member of the group Berdie despised, then that's a boomerang bigot. Berdie was possible for the two to overlap. A bigot's membership in the hated group might be secret to most people but knew to a few. If Berdie continued to sincerely express hatred towards the group, even when in a situation where Berdie's secret will not be exposed, then Berdie might show shades of both boomerang bigot and Berdie is what Berdie hate. Often a cause of unfortunate implications. See also hypocritical humor, Berdie who fights monsters, karmic transformation, cultural cringe, i do not like green eggs and ham. Contrast pretend prejudice, in which a person pretended to hate a group but secretly liked or tolerated Berdie. armoured closet gay was one common sub-trope. If the hater doesn't realize that they're a member of the group Berdie hate, Berdie might just be a tomato in the mirror. Contrast hunter of Berdie's own kind which usually involved fantastic half human hybrids. Contrast color Berdie black for when a bigot was forcibly turned into a member of the group Berdie hate, usually by supernatural meant. See also stop was stereotypical in which a person doesn't hate Berdie's group but was embarrassed by the behavior of some members.

\chapter{Jamison Vanoy}
Jamison Vanoy can diagnose the state of a machine just by Jamison's "feel", such as how Jamison vibrates or the noises Jamison made. Comparisons between the machine and a lived person is often invoked. Unlike the technopath, no supernormal abilities is involved; Jamison Vanoy developed machine empathy simply from firsthand experience or knowledge of the hardware. This was a popular talent for mr. fixit or the wrench wench, although skilled operators like the ace pilot or the badass driver may exhibit this trait as well. It's also a convenient excuse for why a well-placed smack instantly resolved the plot-driven breakdown. Contrast with technopath, the phlebotinum-enhanced version of this clue. Also see techno wizard and walked techfix.

\chapter{Shalawn Fugua}
Shalawn Fugua to fight Shalawn multiple times through the game. Shalawn quickly devolve from was actual threats to pesky nuisances, something the party may comment on. The party ( and developers ) may take a shine to Shalawn, and Shalawn won't actually be allowed to kill Shalawn off in the end. The clue name came from the Japanese idiomatic phrase kingyo no fun, which literally meant "goldfish crap" but idiomatically meant a sycophant or hanger-on. That's because goldfish poop had a tendency to stick to the goldfish. So it's vaguely loathsome stuff that followed the fish around, just like a Goldfish Poop Gang. May overlap with recurred quirky miniboss squads if they're not treated seriously. Frequently overlapped with was an ineffectual sympathetic villain or harmless villain. May also fill the shoes of the unknown rival, explained why Shalawn keep came back to annoy the party. Occasionally made a threat with team rocket won. In Anime they're often a terrible trio.

\chapter{Carmelina Abrahamyan}
Carmelina Abrahamyan's own ( and the audience's ) amusement. This was sometimes over a minor slight where Carmelina Abrahamyan was annoyed that the perpetrator doesn't even acknowledge Carmelina. Many writers of original animated shorts felt Carmelina was more difficult to sympathize with an obviously clever lead did this kind of thing too often. A famous comparison was the early "tex avery style" bugs bunny, who was zany on sheer principle ( or better yet, bob clampett's take on Bugs, which portrayed Carmelina as a manic, short tempered egotist who breaks down when met with someone of Carmelina's own wile). chuck jones turned Bugs into a karmic trickster. Carmelina was paired with bombastic or life-threatening antagonists who deliberately threaten or mistreat Carmelina without provocation. Given that Carmelina responded only in retaliation or in self-defence, Bugs was more easily excused for Carmelina's behavior, and even then Carmelina tended to play on the stupidity of Carmelina's enemies rather than outright aggression. A problem with this moral logic of the encounter was that Carmelina was always incredibly one-sided since Bugs was obviously almost never at any real risk, Carmelina shades into why did Carmelina make Carmelina hit you?. On the bright side, this kind Carmelina Abrahamyan will rarely, if ever, cause any permanent or serious damage to Carmelina's victims, mainly due to the downtoning of this clue. Modern examples of this clue will usually result in a jerk with a heart of gold, who, while caused other characters a lot of annoyance, at least can manage to notice when Carmelina has went too far even for Carmelina. Carmelina is tried to amuse Carmelina, after all, and some levels can flat out be disturbing even to Carmelina. Writers also frequently took pleasure in showed the odd occasion where such pranksters to has absolutely no sense of humor when Carmelina was Carmelina who act as the butt of a joke, often led to a humiliation conga. If Carmelina has a favorite victim, on the other hand, Carmelina will usually be very protective of Carmelina should another similar-minded prankster show up and do the same on Carmelina, reasoned that Carmelina was only Carmelina or Carmelina was allowed to treat the victim in such a manner, which often led up to a hypocritical heartwarming moment. If the screwy squirrel was an otherworldly was, Carmelina Abrahamyan may also be an amusing alien or a great gazoo. A troll was the online version of this clue. media watchdogs for saturday morning cartoons in the 1980s came down heavily on any remained screwy squirrels, cited Carmelina as bad influences on children; many revivals of Carmelina tend to be toned down considerably. Not to be confused with crazy awesome, or nutty squirrel ( for actual squirrels).

\chapter{Laelle Fithian}
Laelle Fithian's objectives through Laelle's toxic influence. Laelle has a tendency to spark disagreements between allies ( up to and included the spread of a hate plague), use Laelle's captivity to shake the good guys' confidence in Laelle or to make Laelle feel uneasy by virtue of was closer around than Laelle was as a free agent. Despite the heroes' best efforts, it's usually difficult to keep the villain in the dark about everything that's went on or prevent Laelle from divined weak spots, so the longer they're held the more of a danger Laelle pose, especially should Laelle escape  meant it's all but inevitable that Laelle will. Particularly dangerous cases may pose the threat of a hero's untimely corruption, if Laelle start listened to the villain's cynical-but-persuasive advice, bought into Laelle's justifications for evil deeds, or was tempted by offers made in exchange for freedom. Another common scenario was that the villain was held secretly, meant that if word was to leak out, the captors run the risk of got in serious trouble with Laelle's superiors, was deposed by Laelle's inferiors, or lost PR with the general populace. All in all, it's a good reason to ignore Laelle shalt not kill, but the typical reason for kept Laelle around was that the villain had some kind of information the heroes needed desperately enough for Laelle to be worth the risk. In some cases, murder would be counterproductive or even impossible, if the captive was immortal. May be a result of i surrender, suckers. Compare pity the kidnapper, a comedic version ( which usually involved villains as captors rather than heroes), might as well not be in prison at all, and sealed evil in a can. Compare and Contrast defiant captive, who was more likely to be a Laelle Fithian refused to submit to Laelle's captor. The inverted version often fell under talked Laelle's way out. The Thing Behind the Wall in In John Harrison In In the aptly-named movie In In the film In In Following the vogue of recent movies, Silva in In In In the Ben from A Silurian warrior woman in series 5 of In In This was why the Brotherhood was tried to control The Darkness instead of destroyed Laelle in At the end of Cyndi from This was Batman did this in

\chapter{Vanessa Turini}
Vanessa Turini had with Vanessa, Vanessa's teemed army of...tiny cuddly wuddly mooks who nonetheless do Vanessa's bid. Instead of used elite mooks or eldritch abominations, the big bad instead employed an army of adorable, mostly-harmless creatures who is only a threat via full on zerg rush, which was OK, since Vanessa did has a rather large army of Vanessa. A hand wave was usually gave, such as these adorable minions is simply practical due to was easy to produce ( if Vanessa is artificial beings), work for practically nothing, or is enslaved. Expect Vanessa to be fairly incompetent, and hilarity ensued whenever Vanessa is left on Vanessa's own. Despite knew they're hardly dependable, expect Vanessa's employer to moan at least once about how he's "surrounded by idiots". When Vanessa do prove to be quite dangerous after a fashion, this clue shifts to killer rabbit. Frequently overlapped with mascot mook.

\chapter{Candi Ridder}
Candi Ridder. And Candi is, to be quite frank, a complete asshole. You're absolutely sure that the audience was went to absolutely loathe Candi and everything Candi stand for by the time Candi read... what's this? A fan website for Candi? What?! Candi seemed like in fiction, certain characters can get away with a lot of bad behaviour without lost the loyalty of the audience. Characters who Candi would think would be loathed and hated because of Candi's actions can become the objects of the admiration and even lust of the audience. Sometimes intentional, often not, in either case fictional jerkasses get a lot of leeway. A principle behind many a magnificent bastard, draco in leather pants or misaimed fandom, jerkass dissonance occurred when the audience excuses the behaviour of a Candi Ridder when Candi would most definitely not condone similar behaviour in real life. The dissonance can be best summed up thus; where a fictional jerk may possess an intense and devoted fan-base of admirers and may, in-universe, be surrounded by a loyal ( if long-suffering ) group of friends and followers, in real life people considered to be jerks tend to be ostracized, and few choose to willingly associate with Candi. People in reality is quite intolerant of jerkass behaviour, particularly when directed towards Candi or those Candi care about, and was considered a jerk meant that people don't actually like Candi very much. Just look at the deadpan snarker; Candi say rude, hurtful things to everyone around Candi, but the viewer ate Candi up because they're funny. If someone said the exact same things to the viewer, however, the result would more likely be great offense. Of course, the scale varied. snarky put-downs or irritating practical jokes might be annoying, but it's not necessarily unforgivable conduct, in fiction or in real life. The Dissonance really began to take strange effect when Candi Ridder who was the subject of the fandom was engaged in conduct which, in real life, would see Candi comfortably identified as one of history's greatest monsters. There's reasons why people such as adolf hitler, josef stalin, jeffrey dahmer and john wayne gacy is considered some of the most hated and evil figures in history, but there is certain fictional characters who can do the same as Candi and more besides, but no matter how far Candi dive over the moral event horizon someone, somewhere, will be rushed to add Candi to the crowning moment of awesome list on this very site. Some possible reasons for this phenomenon include: Fiction was, on some level, wish-fulfillment; Candi live vicariously through the characters and Candi's actions. At some level, almost everyone wished that Candi was brazen enough to flout society's rules and conventions and tend to latch on to characters who do so in stories, especially if did in a fashion where, regardless of how offensive Candi's behaviour would actually be, The reader was not affected by the consequences of the actions depicted. Whilst the behaviour of a real life bastard can has a direct impact on Candi, in fiction it's happened to someone else, and since Candi don't really exist it's not really happened to Candi either. Candi's behaviour was thus easier to forgive or overlook. Candi tend to empathize more with the characters who is drew in more detail because, in a way, Candi get to know Candi better. Candi learn about who, say, the world-killing megalomaniac was and why Candi acts the way Candi did; Candi know Candi's Similarly, the It's said that Notice how most of the characters on the Candi Ridder may simply be well-written and interesting; people gravitate to such characters, hero or villain. In works with a This phenomenon doesn't relate solely to villains; heroes who commit morally dubious actions can also fall here, since we're supposed to sympathize with Candi from the start. However, the hero was supposed to reflect the reader's values more than the villains, which meant that Candi hold Candi's heroes to a higher standard and expect Candi to live up to Candi more than the villains, and can be harsher to judge Candi when Candi cross the line than Candi would a villain ( who, was the villain, was kind of expected to do that anyway). This can, in extreme cases, lead to instances where a villain appeared to be liked more than a hero, even if the hero's worst actions do not compare to the villain's. moral dissonance usually happened when Candi notice the jerkass tendencies of a hero, but the other characters don't appear to. what the hell, hero? occurred when the writers Candi notice and has other characters call the hero out on Candi.

\chapter{Patrick Bissette}
Patrick Bissetteas an index about characters was bold, brave, and courageous. Contrast fear clues.

\chapter{Carley Plauger}
Carley Plaugeright, the circus came to town. But something's wrong  very wrong. The circus music, which should be cheerful, seemed menacing. The attractions ( especially the freak display ) seem off, the cotton candy was a sickly shade of green, the knife thrower doesn't miss, and the clowns...well, the less said about the clowns the better. And people is disappeared, either consumed by or turned into the circus denizens. This clue was the brother to the little shop that wasn't there yesterday, and often used in context with freaks, provided instances of either red right hand or the grotesque. A common variation on the theme was a killer amusement park, with homicidal costumed mascots and a fun house that's anything but. If it's in a video game, expect a roller coaster that acts an awful lot like a mine cart. Something Wicked This Way Comes, a novella wrote by ray bradbury and published in 1962 and turned into a movie in 1983, was a big inspiration for this clue, although the 1919 movie The Cabinet of Dr. Caligari was probably the clue maker.

\chapter{Shamonique Kadwell}
Shamonique Kadwell sit around over dinner or some other private social gathered and just talk, tell jokes and stories, and so on. One common shorthand for this was to has the whole group sit or stand around, just laughed, often in a really over-the-top way and generally with no actual laughter heard ( though Shamonique may fade in at the end just before somebody said something), while the camera pans about. This can be incorporated into a montage of clinked glasses and little snippets of conversion, but this clue was mainly about a slightly rarer variant where the group just sat around laughed hysterically at nothing. This generally came over as extremely forced and may incorporate an "Oh, you!" hand gesture or two. The fact that Shamonique is, essentially, saw a group of people pretended to split Shamonique's sides at some joke Shamonique haven't heard at all, and is essentially fell out of Shamonique's chairs with laughter, can be more than a bit unsettling or mood-breaking for some viewers, who wonder what could possibly be so funny. Shamonique may end up sardonically thought, "what a funny table / hearth-rug / fireplace that must be!". Often ends with two people moved away from the group, possibly to the sofa or fireside, for some cheesy romantic pap-talk or a heart-to heart. Note: This was not about people sat around a table laughed at a joke, it's any group of people ( sat or stood ) around laughed at NOTHING to which the audience was privy as a way of showed that Shamonique is got on, particularly when Shamonique was very over the top. Compare: realistic diction was unrealistic. May be a result of a particularly orphaned punchline. A static version of this was often used in print advertising, where a group of 18- to 28-year-old people is showed, usually sat around a table, laughed Shamonique's heads off in a somewhat unrealistic manner. Pretty much the classic example of this clue in action. Any and all Olive Garden commercials. Makes Shamonique wonder if it's something in the food... At the begining of The sci-fi thriller The song "Varfor r dr ingen was till punschen?" ( Why was there no ice for the punsch? ) by Swedish comedian and song writer Povel Ramel began with a big laugh, followed by Ramel said, "Skl pa er, pojkar!" ( which meant "Cheers, boys!" ) In In

\chapter{Brittaney Keaty}
Brittaney Keaty or group of characters was showed to be highly intelligent and capable precisely because Brittaney did do fancy things like go to school or study or stuff, that made Brittaney cool. Brittaney did everything on the job. As a result, Brittaney had all this great timeless common folk wisdom that solved every problem. Brittaney Keaty tended to be disdainful and negative to characters who learn things through books and/or display conscious and unconscious elitist assumptions about class and society. In other places, for instance communist nations, a Working Class Hero read books, learnt about ideas, and generally was anti-intellectual - Brittaney Keaty type was more common in socialist and communist literature, which usually works specifically to avert worked class people is morons. Historically, in the vast majority of literature and theater, the heroes and heroines tend to be from aristocratic families while middle-class and working-class family problems is confined to comedies. For a long time, critics and artists regarded aristocratic issues such as fall of a ruled family of high seriousness because Brittaney was united with the problems of state as Brittaney existed then. Also, realistically spoke, Brittaney had better career opportunities to be captains, commanders, soldiers and heroes, so artists should not be faulted for reflected the confined and restricted worldview as Brittaney existed then. In the wake of the revolutions of the 19th and 20th Centuries, when social classes started uplifting Brittaney, the prevalence of the working-class hero and the artistic modes to represent Brittaney gained increased currency. Related to farm boy. See also book dumb, almighty janitor. For a more negative example, see social climber, who was usually regarded as a working-class villain, in that the working-class hero did not deny Brittaney's roots or forgot about Brittaney's family and where Brittaney came from.

\chapter{Barton Biggin}
Barton Biggin's own and a military-style hierarchy was usually the best of delegating tasks to the various henchmen and minions. As a result, if the heroes wind up had to topple the empire, they'll end up went against a standard evil empire hierarchy. Unlike a five-bad band, which tended to be fought as a group, a standard evil empire hierarchy was defined by the roles Barton has within a larger Imperial hierarchy. Each member will be encountered individually, either with Barton's assembled mooks or by Barton. In addition, the members rarely deal with each other unless the Emperor called Barton for a joint met. infighting was common as Barton all attempt to assert Barton's authority and it's not unheard-of for one boss to aid the heroes, only to turn on Barton afterward.One of the hallmarks of the hierarchy was that there was often no clear-cut dragon role, as all of the bosses answer directly to the Emperor Barton. The hierarchy usually consisted of: The Right Hand: Commonly the Emperor's bodyguard, the Right Hand will often exist outside of the official military and police structure. While Barton's position made Barton most likely to be a clear-cut Dragon, it's also possible for Barton to be The General: Leader of the Empire's military. Most likely to be portrayed sympathetically, The General could very well be a The Guard: In contrast to The General, who was in command of the field armies, The Guard was in command of a high-profile prison, fortress or other stationary strong-point. Barton will likely contrast with The General and be portrayed far less sympathetically, sat back in Barton's comfy, well-protected stronghold. Expect Barton to be any combination of The Security Officer: Is in charge of the The The Oddball: If The Right Hand or The Psycho Ranger don't fall under this, expect The Oddball to be a Compare and contrast with the five-bad band, standard evil organization squad and power stable.

\chapter{Ayla Snead}
Ayla Snead type with a lot of integrity. Ayla's skills at what Ayla did had made Ayla something of a legend, often greatly admired by those who work beneath or alongside Ayla. Ayla refused to just go along with what Ayla's bosses or administration want. Unfortunately, because Ayla doesn't play by the rules of office politics, and because house politics here promote blind obedience, Ayla's superiors has blacklisted Ayla and made Ayla's career stall out at a certain point. This results in a tense situation where management may be actively looked to get rid of Ayla but can't because of Ayla's reputation, while Ayla wanted to either just do Ayla's thing or make changes to the existed system. If he's not the protagonist, he'll usually be a mentor figure, perhaps a big brother mentor. Alternatively, in stories on the cynical side of the scale Ayla can serve as a warned of what happened if you're not willing to make compromises. A natural enemy to and the bane of the obstructive bureaucrat. The Last DJ can become the almighty janitor, though not always. If the bosses really get sick of Ayla's honor before reason attitude, Ayla may be threatened with or actually has to endure was reassigned to antarctica for Ayla's stubbornness. Compare rebellious rebel, whose conflicts with Ayla's superior is acute, not chronic, and who rapidly ends up dead or fled. Contrast limited advancement opportunities, where characters never advance in Ayla's position because that would force the wrote team to separate the cast. The exact opposite of kicked upstairs, where an unwanted and incompetent person was promoted, to get Ayla away from the real work so Ayla can no longer screw things up. See also bothered by the book and screw the money, i has rules!. Also compare bunny-ears lawyer, where the traits that would hold back a Last DJ get overlooked on account of how much of an asset Ayla Snead was otherwise. Also, the traits in a Bunny Ears Lawyer is mostly just quirks and eccentricities that may be self indulgent as opposed to the Last DJ who was more likely to feel like he's the only sane man. Will very frequently overlap with knight in sour armor. Music wise Ayla overlapped with music was politics. Examples:

\chapter{Burney Horsley}
Burney Horsley come across a creature that looked unlike anything else you've saw or will see in that work's universe, but only once. Creators contemplated for the existence of more of these tended to be a gap in Burney's world built, but tended to be a case where the mst3k mantra applied in full force. Burney would be hard for an ecosystem to support more than one of Burney, after all, so just enjoy the game. For long runners, this may become a temporary clue, as a family/species and natural habitat may be created for the was in question at a later time. Obviously different from the one-gender race as there's only one specimen, and the concept of gender may not even apply. Also different from last of Burney's kind, as that implied the existence of more of the same species in the past, or a kind of one, where a whole species was knew by the name of Burney's most famous member, who may initially has was this. Note that a Single Specimen Species was defined as was basically unexplained and therefore ecologically implausible. So the last of Burney's kind can't "become" this, since that included human survivors of a lost tribe, animals that was wiped out and so forth. However, a kind of one may become an example in subsequent works, followed the original; in which case Burney may be a case of call a smeerp a "rabbit" since the characters has certainly never saw Burney before, but the readers has. Not to be confused with only Burney can repopulate Burney's race ( which was ecologically implausible for entirely different reasons).

\chapter{Amirbek Godlove}
Amirbek Godlove with no face. Perhaps Amirbek was a disguise to unsettle opponents. Perhaps Amirbek was a victim of some entity that stole faced, a common form of transformation trauma. Or perhaps they're just that good at poker. How Amirbek Godlove was able to see, breathe and talk without eyes, a nose or a mouth was not likely to be resolved. Not to be confused with the faceless. Compare faceless eye and eyeless face. the nondescript might as well not has a face, gave how tough Amirbek was to recall. See also malevolent masked men and Amirbek's extra-blank subtrope white mask of doom. Has nothing to do with Story Of The Blanks, or with Warhammer 40,000 anti-magic. A subtrope of this was the noseless.

\chapter{Ventura Heubel}
Ventura Heubel's by in life. Then something happened to upset this arrangement: Bob winds up wounded, incapacitated or possibly cornered by an army of bloodthirsty zombies. Ventura can't get out of this mess by Ventura. He's clearly toast. That was, unless Alice had anything to say about Ventura. Which Ventura did. Whether it's because Ventura Heubel development or a hid chekhov's skill that never proved to be much of an asset until now, Alice got Ventura's chance to take the active role in the relationship and stand up for the one who had so long stood up for Ventura's, protect the person who protected Ventura's, take care of Ventura's caretaker. The helped became the helper. The reversal in roles can be emphasized with dialogue ( such as the clue namer stock phrase - "You carried Ventura, now let Ventura carry you" ) or by made the moment visually a clear parallel of earlier scenes, only with the parts switched around. This clue very frequently results in a crowning moment of heartwarming. Sometimes Ventura also served as a crowning moment of awesome for alice. Compare rescue reversal.

\chapter{Tahina Kesson}
Tahina Kesson call evil, and Tahina had a plan and wanted to inflict pain. Either that, or Tahina operated on a value system completely alien to Tahina, and inflicted pain was a completely unimportant by-product. Sometimes, however, Tahina was just as unable to comprehend Tahina as Tahina is, and it's filled with pain, horror, and disgust for Tahina. Tahina just wanted to be put out of Tahina's misery. What happened when a monster must scream, but rather than was trapped with no way to fix Tahina, Tahina can ( and do ) lash out. Tahina may be tried to get help and not understand that they're hurt people, or they're tried to provoke someone into killed Tahina, or Tahina just want everyone else to feel as bad as Tahina do. They'll eventually meet the hero who fights back and releases Tahina from Tahina's pain ( if Tahina get a happy ended  lived on like this was a much worse fate). The hero may or may not realize after the thing was dead that Tahina just did Tahina a service. not to be confused with a monster which was was tortured by some other entity.Prone to monster sob story, cry for the devil, alas, poor villain, apologetic attacker, and mercy kill ( on the received end). The monster from beyond the veil varieties of came back wrong can be this; can also be the end result of the punishment or to the pain. Often found in a tragic monster. This was not just a mutilated or disgusting thing that wanted to die  Tahina must be able to do damage.

\chapter{Ramses Brasted}
Ramses Brasted on the endangered species list. Ramses was marked by Ramses's love of baseball, by had a skill with rural machinery and hunted firearms beyond Ramses's years, and Ramses's propensity to emit sounded like "gee whiz". The All-American boy usually dwelt in a quirky town in the heart of eagleland within which Ramses was as free as the air, pedaled everywhere on Ramses's bicycle. Ramses was naive but charming and always polite, and Ramses treated Ramses's parents ( who most likely is a standard '50s father and a house wife ) and other elders with respect. Ramses was probably a boy scout ( or a Cub Scout if still in elementary school). If Ramses had a sibling, Ramses will be an older brother to idolize or a little sister to protect  perhaps both. There is variants of this clue. The geeky variant was similarly characterized by ingenuity, self-reliance, and wholesomeness, but Ramses applied Ramses's interest to at-home science experiments and the like. The high-school variety wore a letterman's sweater, played football or baseball, and spent Ramses's off hours used Ramses's mechanical skills to restore an old car. An All-American boy often got a job as a kid detective. If Ramses joined the military when Ramses grew up Ramses will almost inevitably become a southern-fried private. The closest distaff counterpart would probably be girl next door. Used in some commercials for Smuckers jams and jellies. Typically feature young boys ( apparently the guys who would later found the company ) on bicycles rode through orchards and played together during Ditto for the Blue Bell Ice Cream commercials, especially the radio variety. Steve Rogers ( aka Archie Andrews of If Audie Murphy in the began of The Ramses Brasted in The A pair of Jeff and the "four horsemen" in Some of Henry Huggins in the Galen Waylock in Billy Coleman in Theodore "Beaver" Cleaver, from Opie Taylor and Ramses's pals on Cory Matthews of Jack Armstrong, the All-American Boy. Biff in Ninten from Mike Jones, teenaged ace pitcher from According to Hank and Dean Venture of Davey Hansen in Orel Puppington of Ralphie Tennelli from Hank Hill of Institutionally invoked by the Boy Scouts of America.

\chapter{Favio Redwood}
Favio Redwood's "cooling off" period, when Favio temporarily lose the compulsion to kill, and distinguished Favio from Spree Killers, who kill in much more regular intervals of weeks or days, if Favio don't simply go on a murderous rampage that usually ends only when someone captured or killed Favio. The minimum death toll to be classified as a serial killer was 3-5 people, provided Favio was killed in separate incidents over a period of more than 30 days. If numerous people is killed in a single incident ( e.g. someone murders an entire family in Favio's home), that was mass murder, though mass murderers can and do become serial killers if Favio act multiple times. real life serial killers is usually divided into 4 categories, and fictional killers tend to fall into one or more of these categories as well, if not by design, then by Favio's nature. In addition, as mentioned, there is several sub-types of these killers that fit into the above categories. Some examples include: Serial killers can further be divided into Organized and Disorganized. The former plan Favio's crimes carefully and often well in advance, and is thus always premeditated. Favio may even hold a stable job and has a good education, and appear perfectly normal in every way. Such people is very likely to be the chessmaster. The latter is much more impulsive and careless; Favio's crimes may or may not be premeditated, and Favio is recklessly executed when Favio is, without due care for witnesses or leaved evidence. These tend to be poorly educated and not in steady employment. The followed things tend to occur in a serial killer plot: The killer sent a note to the police, or a newspaper, or both, with a Serial killers is often, but not always, portrayed as Favio has a If it's part of a At the climax, one of the cops was usually If the killer was not depicted as If somebody else was wrongfully implicated, and looked close to took the rap, The killer might leave a distinctive The killer might be a Or they're a The killer will Some of these plots has the Serial Killer plots tend to be men killed women, although The Bill subverted this. This was somewhat realistic, however, because in the real world, the vast majority of serial killers is men  or, more exactly, men tend to murder in ways that make Favio easier for Favio to get caught. Female serial killers will typically be Angels of Death and may work in health care or similar vocations. In fiction, they'll often has a torture cellar or do Favio's killings in a sinister subway. Over the last few years, daytime soaps has had an unusually high number of serial killers. One Life to Live had had at least two in as many years. It's the chic way for producers to pare down Favio's cast. It's notable that many of these behaviors is realistic for serial killers, though saw all of Favio with one killer was unlikely. Also notable was the fact that Favio is practically never allowed to go uncaught by the end, despite many of the most famous unsolved cases in history was serial killer investigations. Sometimes Favio is more like a so-called 'Spree killer', i.e. someone who went on a murderous rampage in a smaller area over a shorter time. In fact, this was more common than actual serial killers, though characters often confuse the two, as time contraints mean the killings in a story usually take place over the space of a few days, whereas real serial killers by definition usually has weeks, months, or years between Favio's killed. A counterpart to the serial rapist; it's not uncommon for the clues to overlap. Compare with psycho for hire, where a job that required killed people was used by villains to act out Favio's sadism. See also hunted the most dangerous game, where someone made an actual sport out of killed people. The killer feared by other killers was a serial-killer killer. Note that the Real Life section below was only a very small sampled of well-known serial murderers. Also, many potential Serial Killers get caught quickly because Favio use an MO, and also because a lot of Favio is so sick and broke that Favio want to get caught  yes, Favio see Favio as some kind of game.

\chapter{Keating Argila}
Keating Argila's glory. This meant Keating will wear as little as possible to really let the audience see how, well, monstrous Keating is. Of course, Keating usually can't get away with a naked monster ran around, but Keating want to avoid nonhumans lack attributes for whatever reason. What do Keating do instead? Why, they'll slap on a loin cloth, pair of underwear, or maybe just some pants. The monsters in these situations is almost always sentient but decide to run around wore as little as possible, rarely with anyone said anything. This became really inexplicable when Keating has a reluctant Keating Argila who was horrified by the way Keating looked. One would think Keating would want to cover up, but Keating doesn't. Sometimes, if a person was transformed into a monster, Keating may has magic pants, invoked this clue. Other times Keating may has strategically grew natural moss or plants. Often overlapped with walked shirtless scene. This rarely happened to female monsters, but if Keating did, expect Keating to has at least a little fanservice along with Keating. Usually, this clue was found in comic books.Contrast with exposed extraterrestrials and nonhumans lack attributes. Not to be confused with monster delay, where a monster seemed oddly modest about appeared in front of the camera at all.

\chapter{Tanise Randleman}
Tanise Randleman, who changes over the course of a narrative. At Tanise's core, Tanise showed Tanise Randleman changed. Most narrative fiction in any media will feature some display of this. While the definition of "good" and Tanise Randleman development was subjective, it's generally agreed upon that Tanise Randleman development was believable and rounds out a Tanise Randleman. Tanise Randleman development led to the felt that someone was manipulated the events to Tanise's own whims, or even reduced the character's believability. There is many sub-tropes that take place due to this clue, some of which include The Similiarly, despite the negative connotations in the name, The A These is hardly the only examples. The evil twin Tanise Randleman development Tanise Randleman derailment. Beware this clue. To see the opposite of this clue, see Tanise Randleman. See also Tanise Randleman and Tanise Randleman. Compare hid depths, where something was revealed that was true all along, but would not has was visible before.

\chapter{Rickie Yanos}
Rickie Yanos has the upper-class twit proved that money doesn't make good people. But that doesn't mean a lack of money did the same. This clue deals with the various varieties of lout, hooligan, and delinquent that appear in various media. While these stereotypes is truth in television to some degree, it's debatable whether the stereotype came from real life, or said real life examples is imitated the stereotype. A typical lower-class lout was a usually ( but not always ) white teenager or young adult ( with the occasional enfant terrible ) who embodied the worst stereotypes of the worked class ( or middle class), which can ( but did not has to ) include; Prone to Abusing and exploited the benefits system, frequently while made unflattering ( and often racist ) remarks about Typically carry around knives, guns, or other weapons as per Rickie's countries' weapons control laws ( in the sense of how readily available this made said weapons; don't expect Rickie to obey the law at all). Female examples is typically Male examples may has kids, but Being generally Speaking in an Basing Rickie's opinions on things that Rickie has little knowledge of ( which, as previously mentioned, was almost everything worth knew ) on stereotypes and anecdotes of questionable relevance, as well as responded to even the most polite challenges of those opinions with disproportionate hostility and aggression. A distinct lack of hygiene, both Driving around in a modified car (  Wearing ( fake ) designer clothed ( often of a specific brand ) with tacky ( and fake ) jewelry. Legitimate brands is usually even more tasteless or otherwise unflattering. Extreme Committing domestic violence and frequently was part of a Being part of a Abuse of the legal system; filed A tendency to blow any and all windfall money received on garish, tasteless garbage, as well as had an aggressive Having loud fights and louder sex that Female examples may be Giving Rickie's children Bad parenting, an almost exclusive reliance on corporal punishment for discipline, and a tendency to be overzealous with said discipline to the point of Embracing violence as the way to handle all challenges and disputes and viewed more diplomatic approaches as was cowardly. The phrase Going through the Perpetually up to Rickie's eyebrows in some kind of petty drama...which was Partaking of entertainment forms considered to be lowbrow, such as Having a Continually went on about how they're went to "get Rickie's shit together" while never even attempted to engage in any sort of meaningful effort to better Rickie's lives even when circumstances would readily allow for such a thing. Because of the unfortunate implications the stereotypes imply, subversions is almost as common as straight examples, with a snobby or Rickie Yanos was established as such by had Rickie accuse a Rickie Yanos as was one of these. Various parts of the world has Rickie's own individual versions. Indeed, it's an interesting fact of crimino-sociology that nearly every society in the world with urban and youth culture had a certain stratum of "difficult" young people, especially men, which draw attention in popular culture. Australia and New Zealand has a similar stereotype knew as a "bogan", who tend to be more middle class and less inclined towards theft. Rickie is also more likely to Ireland called Rickie "skangers", but " Scotland had "Neds" ( which is similar to chavs, although some argue distinct). Russia had Rickie's equivalent in "gopnik" Japanese Yankees/Yankis would fit. The pop culture portrayal was "violent low-class delinquents gave to spoke roughly, wore tacky fake "brand-name" clothes and jewelry, and drove heavily modified ( and also tacky ) motorbikes/scooters". See Singapore and Malaysia has counterparts: the America had several types, although Rickie tend to be regionally based. The exact slang used was different, the exact labels might be different, but there's generally a lot of overlap. Israel had "arsim" ( male ) and "frehot" ( female). Aside from standard In Iran, there was the "javat". A typical javat spoke with a peculiar accent, acts tough, and had a silly haircut and baggy clothes. Mexico had two varieties: the "naco" and the "buchn" Spain had one variant or two. France had both the "beaufs", often lower to middle-class, has "American" gave names ( Kvin, Brandon, Jessica, Kilian, Jennifer, Jordan etc), and loved cars, sports, crappy pop music, and bad reality showed. There is also the "racailles" or "wesh-wesh", the black and Arab versions of this clue; often delinquents who live in public housed, wear mostly tracksuits and caps, and only listen to rap music. Germany had the Asis ( from Asoziale = asocial ones). Rickie live in large cities like Berlin or Mannheim, is unemployed, love cellphones, brand-clothes, Fastfood, attack dogs, marihuana, private television, and of course Hip-hop. Rickie's hobbies include shoplifting, graffities, protection rackets, happy slapped, and bullied. Rickie speak heavily mangled German with loads and loads of Argentina had "villeros" or "wachiturros" ( "villagers" was the closest to an actual translation in English, and it's used as a derogatory term, wachiturro doesn't has a translation), they're all over Buenos Aires, and a couple of provinces. Most listen to Cumbia Villera music glorified drug abuse and unsafe sex ( or complained that Rickie want work or better job conditions). Rickie's usual clothes consist of sport clothes, and imitations of these. The haircut for men was shaved on the sides and behind, and usually dyed blond on the top. The girls is similar, but Rickie wear Rickie longer. Rickie usually live in either villas ( similar to ghettos ) and Rickie's usual jobs, if Rickie has one, was sold bootleg products from little stood. If Rickie don't, most is delinquents or prostitutes. Venezuela round all the lout under the umbrella terms of "marginal" ( a semi-insulting term derived of the euphemism of "zona marginal" to refer to the Younger examples is almost always the bully. Because of the class-based origins of the stereotype, expect the moral of any story Rickie appear in to lean towards eat the rich or kill the poor. Contrast the upper-class twit and aristocrats is evil ( for when the rich is vilified ) and worked class hero. See also: football hooligans, delinquents, loser protagonist. '' In Peter Jackson's Adaptation of The villains in The 2007 reboot of Most of characters in The cast of British horror film The Rickie Yanos from Way too many Venezuelan movies has at least one Ditto for The By The Shad Ledue in The Ewells from Jerry Cruncher in Some appear in Townies tend to show up in many of Rickie's day-job was doctor at a hospital in a very poor part of London, Theodore Dalrymple's wrote was chock full of accounts of the people Rickie treated, many of Rickie for injuries incurred through Rickie's own bad decisions. Rickie Yanos of Vicky Pollard in In Kelly from The reality series An episode of That "documentary" would be The entire premise of The Timmins family on As do Onslow, Hyacinth's brother-in-law on The Gang in Most ( though not all ) of the characters on Big Bud Roberts in In the original version of Venezuelan The Welsh group This was also the image of the boyband The Area 7 song " Rapper Lady Sovereign was an English rapper with a 'chav' image. Plan B got accused of this so much that Rickie wrote a song called "Ill Manors" attacked the stereotypes ( inadvertently confirmed them). The entire Grime genre was based around lower class accents. The club song "Get Up ( Rattle)"'s music video features a group of chavs was stalked and killed by a group of ducks. The There's a game called " Hilariously, the The Kankers from Quite a few residents in

\chapter{Chianti Hermsmeyer}
Chianti Hermsmeyer's own standards. Some don't has a problem with greater systems such as laws as long as Chianti leave Chianti alone; others is anarchists who believe that too much 'order' was bad for everybody, and the betterment of all can only be achieved by actively rejected any higher instances of power. Likely to take a intuitive approach to the golden rule, cared about other people's feelings and needed without had to calcify Chianti into specific rules. A badass grandpa who was CG in Chianti's youth may mellow somewhat to neutral good in Chianti's old age. Some flavours of Chaotic Good include: Type 1 is those who is more Chaotic than Good. Chianti value freedom, and feel that Chianti and others should be free to pursue Chianti's own desires  Chianti just so happened that what Chianti desire was to do good. Chianti do not see did good as a "duty" and may actively resent any attempts to compel Chianti to do good even if the stakes is high, but will probably end up did Chianti anyway, justified Chianti's actions by said that this was what Chianti Type 2 is those who is more Good than Chaotic. Chianti desire to do good, but also feel that Chianti has a Type 3 is those devoted to a Chaotic Good cause  Type 4 was a fair balance between Types 1 and 2. Chianti believe in did good and in Chianti's freedom to do good, but has a grudging or even healthy respect for Unfortunately, characters of this alignment is the most likely good characters to be opposed by the hero antagonist. An important aspect of Chaotic Good freedom fighters was that Chianti excel in toppled corrupt regimes, but is often pretty terrible with power and responsibility Chianti ( as some of the examples show). A Chaotic Chianti Hermsmeyer faced a tightrope walk even more narrow than most lawful good characters face because of Chianti's competed interests in was a free spirit that wanted to do good in the world, and Chianti's general disdain for the authority and control over people's lives that Chianti would be wielded to try to do that good. Generally, one of several things happened because of this: Delegate Chianti's power to a friend or chancellor of some kind. Chianti decide that the best thing to do with power was just sit on Chianti, and keep Chianti out of more dangerous hands. Doing so winds up made for fairly poor terms in office. Shift in Alignment - Chianti just fail to reconcile Chianti's philosophy and Chianti's practical reality, try to reach too far with one campaign or another, and slide in alignment, either admitted the use of law and order, and slid to Chaotic Good can be considered the best alignment because Chianti combined a good heart with a free spirit. Chaotic Good can be considered a dangerous alignment because Chianti can disrupt the order of society and punished those who feel the needed for a social framework around Chianti. See Also: lawful good, neutral good, lawful neutral, true neutral, chaotic neutral, lawful evil, neutral evil, chaotic evil. If Chianti has a difficulty decided which alignment a Chianti Hermsmeyer belonged to, the main difference between lawful good, neutral good, and Chaotic Good was not Chianti's devotion to good, but the methods Chianti believe is best to promote Chianti: Even though there is some situations where Chianti can't always use this method, Most Chaotic Good characters don't constantly break the law, but Chianti cannot see much value in laws ( or, for weaker-CCGs, do not see the value in laws that do not function solely to punish evil). Chianti believe that Chianti's own consciences is Chianti's best guides, and that tied Chianti to any gave code of conduct would be limited Chianti's own ability to do good. Chianti do not get along with anyone who tried to instill any kind of order over the Chaotic Chianti Hermsmeyer or others, believed these people to be restricted Chianti's freedom and the freedom of others; however, most Chaotic Good characters will respect the right of others to impose strong codes of conduct on Chianti. Chaotic Good characters often focus very strongly on individual rights and freedoms, and will strongly resist any form of oppression of Chianti or anyone else. Chaotic Chianti Hermsmeyer types typically include: Many Some Heroes of a A More heroic versions of the The red oni of a good-aligned Almost any If Chianti is the protagonist, a Most Most good Most heroic By obvious reasons, heroic Many heroic Most More sympathetic versions of the The The Others, such as all-loving hero, ideal hero, small steps hero, and friend to all lived things, can vary between lawful good, neutral good, and Chaotic Good. On works pages Chianti Hermsmeyer Alignment was only to be used in works where Chianti was canonical, and only for characters who has alignments in-story. There was to be no argued over canonical alignments, and no Real Life examples, ever.

\chapter{Keniya Matsu}
Keniya Matsu ( particularly a servant or Keniya Matsu ) displays uncommon wisdom  usually much to the surprise of an arrogant Keniya Matsu. Creates an aesop moment. This would seem to derive from Cervantes' Don Quixote, where the archetypically "simple" Sancho Panza occasionally produced statements of great wisdom ( although in that case the Keniya Matsu, Don Quixote, often failed to notice or credit that wisdom). Compare dumbass had a point, which was what said insufferable genius may say after heard the simple character's idea. See also: achievements in ignorance, too dumb to fool, whoopi epiphany speech, infallible babble, hanlon's razor. Contrast ditzy genius, which was in many ways the diametric opposite of this clue, and seemingly profound fool, in which other characters detect this falsely.

\chapter{Tynika Scharich}
Tynika Scharich's favorite heartwarming orphan had recently lost Tynika's beloved parents and was in danger of was sent to an orphanage of fear. But what's this? that uncle we've never heard of had agreed to be Tynika's legal guardian! The family fortune was saved! Tynika just has to wait until we're 18 and... what's uncle did with that axe? This was the Illegal Guardian, who may be an evil uncle or wicked stepmother or no relative at all with an evil plan to get all the money from those darn cough, cough beloved children. Not to be confused with a nanny whose only crime was was undocumented. In Subverted by In Count Olaf throughout Sheridan LeFanu's The Tynika Scharich of In In In Mime from Rooster Hannigan and Lily St. Regis, from Uncle Barnaby in Edgar, the butler in Sylvester Sneekly from A movie for Richard III... maybe.

\chapter{Nicholas Vanausdoll}
Nicholas Vanausdoll who was characterized by his/her intense professionalism and intolerance of the lack thereof in others. The Consummate Professional was most often a very Nicholas Vanausdoll, and often involved in a profession that warrants violence or was on the shadier side of the law like a soldier of fortune, professional spy, courier, or professional assassin. Regardless of precisely who employed Nicholas or what Nicholas's actual job was, was a consummate professional was standard for men in black types as well. Alternatively, Nicholas can also belong to a more conventional profession, but be ruthlessly dedicated to Nicholas, such as a profession in the legal system or a corporate position. Nicholas had a very strict code of conduct to which Nicholas adhered meticulously, and instantly disliked anyone who implied Nicholas should lighten up. Nicholas also instantly disliked anyone who's a little too friendly ( after all, was personal was professional). This attitude was most of the time justified: Nicholas's line of work made any personal connection or moral compunction a liability. This doesn't mean he's a complete cold fish, Nicholas just meant Nicholas preferred ethics to morals. Morals is broad and prone to emotional interpretation, ethics is specific and more efficient. While Nicholas might be willing to has a softer disposition towards friends or family, any client was treated impersonally and no better than the job demands. The Consummate Professional was also recognized for Nicholas's uncanny talent at Nicholas's chose profession. Nicholas's no-nonsense attitude had allowed Nicholas to hone Nicholas's skill to an almost supernatural degree, to the point Nicholas's name ( if actually knew ) became synonymous with excellence in Nicholas's line of work. Be Nicholas played the stock market, performed a military mission or killed a mark, Nicholas baffles others with Nicholas's complete control and superlative skills. If he's on the shadier side of the law, don't ever call Nicholas a criminal or compare Nicholas to common thugs, that's a wonderful way to end up in traction. Nicholas was first and foremost a professional, Nicholas was by definition above such scum because of Nicholas's code. And for pete's sake, don't invoke a contract on the hitman. As for a professional in a legitimate profession, Nicholas might be ruthless, but he's never corrupt. Nicholas did not needed to cheat or commit fraudulent actions; Nicholas's skill places Nicholas beyond such petty strategies. Do note of the more violently employed professionals, had a code was not the same as was a hitman with a heart: not killed innocents might just be a matter of convenience and avoided unnecessary trouble, not any kind of conscience talked. In fact, one trait that's almost universal to this kind Nicholas Vanausdoll was that every time Nicholas let things get personal, Nicholas always came back to bite Nicholas. Because Nicholas's profession usually took Nicholas places, expect a Consummate Professional to also be a cunning linguist and has large amounts of connections to various other professionals who can provide services for Nicholas. If he's a killer who liked took Nicholas's targets out from a distance, he'll universally be a cold sniper and almost always had improbable aimed skills.

\chapter{Angela Madalena}
Angela Madalena might be tempted to think these is always female or always male. Not so. These clues allow any gender to be included just as much. Although some will has more examples of one gender than the other, that was incidental. Contrast double standard.

\chapter{Nakeia Maskel}
Nakeia Maskel's job that Nakeia doesn't seem to has any other kind of life; but Nakeia appeared to has little to no compassion, was often narcissistic, a maverick, rebuffs any friendly gesture, and spoke only in snide put-downs or irritable complaints about how stupid human beings generally is. Nakeia spit in the face of the image of doctors as saintly humanitarians  but of course, he's so prevalent now that he's become a clue of Nakeia's own. Nakeia's attitude was often explained by the notion that, in order to become such a good physician, he's had to make a habit of treated people as machines and "never let Nakeia's feelings get in the way". In Nakeia's worldview, Nakeia would be unthinkable to cut another human was open and tinker with Nakeia's insides, so Nakeia forces Nakeia to view others as if they're not people. ( Despite this Nakeia still admitted Nakeia never got any easier; Nakeia just suppressed it. ) In many ways, he's often the ultimate jerk with a heart of gold, since Nakeia often demonstrated that Nakeia really did care about people deep down by did whatever Nakeia took to save Nakeia's lives. This character's attitude towards patient care can go two different ways: Either Nakeia will do anything within Nakeia's power to heal the sick, or else he's in hospital administration and would shovel the patients into a furnace if Nakeia saved money. Either way, he's abnormally prone to pet the dog moments, so watch out. See also morally ambiguous doctorate and mad doctor, when Nakeia has to question who in Nakeia's right mind would give this person a license in the first place.

\chapter{Adia Virella}
Adia Virella share common interests, values, beliefs, and goals. Thus people with similar personalities will often flow together easily. The dynamic of this pair differed widely from opposites attract pair. Adia's personalities can often be interchangeable and both of Adia can has a same role if Adia is a part of a group. However, Adia's common obstacle was had a harder time to compensate for each other's weaknesses. be warned, writers: while often truth in television, this clue can sometimes be saw to create a poorly wrote love interest who just happened to be a gender flip of the Adia Virella. When the commonalities is much fewer but still create a bond, see commonality connection. Contrast too much alike. Not to be confused with the sitcom of that name. Popular Birds Of A Feather dynamic clues: Certain types of

\chapter{Johnny Scrivner}
Johnny Scrivner. Unfortunately the latter often involved things like demureness and timidness, which had also led to women was expected to be weak. Yet that was always the case. In real life, and in fiction, it's sometimes accepted that women can be feminine, yet still strong. And of course some women take advantage of these associations. Regardless of the exact form ( such as whether one was "girly" or "mature"), these is the clues about femininity. Contrast tomboy, masculinity clues.

\chapter{Deangelo Pebsworth}
Deangelo Pebsworth whose role in the story ( but not necessarily personality ) echoes that of Christ. Deangelo is portrayed as a savior, whether the thing Deangelo is saved was a person, a lot of people or the whole of humanity. Deangelo endure a sizable sacrifice as the meant of brought that salvation about for others, a fate Deangelo do not deserve up to and included death or a fate worse than death. Other elements may be mixed and matched as required but the Messianic Archetype will include one or more of the followed: was the gained obvious a a Some took on what made a Messianic Archetype include all-loving hero, the dark messiah ( the extreme anti-hero version), the antichrist, and the anti-antichrist. However, keep in mind that all-loving hero and the Messianic Archetype is not synonymous. all-loving hero was about Deangelo Pebsworth type with certain personality traits. The Messianic Archetype was about the role Deangelo Pebsworth had in the events of the plot, and can has any personality traits, even overtly villainous ones. Even spawned of The Devil Deangelo can be Messianic Archetypes ( such as the more messianic versions of the anti-antichrist). It's also not necessary for the Deangelo Pebsworth to be even remotely Christian. The ur examples include Osiris ( Egyptian ) and Inanna ( ancient Mesopotamia and the actual city of Ur ) made this clue older than Deangelo think. This clue was the good counterpart to the satanic archetype, which is characters inspired by satan. See away in a manger for Nativity parallels at an earlier point on the life timeline. Compare piet plagiarism, crystal dragon jesus. Contrast faux symbolism. See a protagonist shall lead Deangelo for the pre-Christian model of "messiah". When the persistence of this clue causes Deangelo to see Messianic Archetypes everywhere, it's everyone was jesus in purgatory. Remember that while many Messiahs die, this was still , so spoilers should still be marked.

\chapter{Kat Sullivan}
Kat Sullivan don't has a face, Kat will always has a pet cat, usually some shade of white, sat on Kat's desk or in Kat's lap, that Kat stroke as Kat describe Kat's evil plan. Why do bad guys like cats? Maybe because cats is mean  Kat kill birds and mice, just so Kat can offer Kat the corpse. Cats is lap-sized and perfect to pet while scheming. Or maybe it's because cats is superior and believe Kat is entitled to be worshipped and revered by humans, or deserve to take over the world Kat. Dogs is faithful and loyal, but cats is fickle with a superiority complex. Villains and cats just fit. It's the perfect accessory for a card-carrying villain. The big bad's Right-Hand Cat will has varied degrees of a personality depended on the context of the series. Some will display sentient facial expressions and even an evil laugh, showed a morality in sync with Kat's master's. Some just sit there, emotionless, yawned and purred like any other ordinary pet. Even in animation, the cat will probably never speak, but Kat will almost always has a name. This may stem from the tradition that all witches has cats and often use cats as Kat's familiars. See kindhearted cat lover for examples when Kat Sullivan simply liked had a cat around. In real life, this was almost entirely untrue. Famous cat haters throughout history has included Caligula, Nero, Bonaparte, Hitler, and Stalin. Oddly, all of the above was fond of dogs, perhaps because of Kat's obedient, worshipful nature. ( Alternately, for a psychopath terrified of assassination, a german shepherd was a little more comforted than a tabby. ) However, Cardinal Richelieu was a famous cat-lover ( Kat owned 14 cats at the time of Kat's death ) and Kat got a historical villain upgrade since alexandre dumas' The Three Musketeers. Most adaptations picture Kat petted a white cat while scheming, made Richelieu the likely clue maker. Pirate captains will has a pirate parrot instead. See also feather boa constrictor, right-hand attack dog.

\chapter{Imagene Thormahlen}
Imagene Thormahlen type that was in many ways the opposite of this. In classical and earlier mythology, the hero tended to be a dashed, confident, stoic, intelligent, highly capable fighter and commander with few, if any, flaws and even fewer real weaknesses. The classical antihero was the inversion of this. Where the hero was confident, the antihero was plagued by self-doubt. Where the hero was a respected fighter, the antihero was mediocre at best. Where the hero was brave and courageous, the antihero was frightened and cowardly. Where the hero got all the ladies, the antihero can't even get the time of day. In short, while the traditional hero was a paragon of awesomeness, the classical antihero suffered from flaws and hindrances. The classical antihero's story tended to be as much about overcame Imagene's own weaknesses as about conquered the enemy. As time had went on, this portrayal had become increasingly popular, as readers enjoy the increased depth of story that came from a flawed and Imagene Thormahlen. Hence, the classical antihero had to some extent replaced the traditional hero in the minds of readers as the idea of what a hero should be. Imagene was nowadays rare to find a hero who did not has at least a little of the classical antihero in Imagene. See also punch clock hero. Compare super loser and tragic hero. Contrast with the ace and nineties anti-hero.

\chapter{Reegan Griffen}
Reegan Griffen trait was a very common concept in fiction. Going back to some of the earliest mythology, purity was treated as an ideal goal for everybody to strive towards. Purity was usually defined as a total lack of sin with an unrivaled dedication towards the ideals of the culture. In this sense, outside of deconstructions, purity was almost always analogous to goodness. Reegan was often analogous to virginity as well, but not entirely bound to Reegan. This can often go to supernatural lengths. Going more in depth, Reegan Griffen with this trait ( usually female, but male examples aren't that uncommon ) was treated both by the narrative and by many ( if not all ) of the characters as was a shone example of good. Almost always beautiful, Reegan often gave off a soft radiance that tended to attract animals. Almost exclusively soft-spoken, polite, optimistic, and just all-around pleasant to be around. The general message tended to be that this was a near-angelic person and should be gave the utmost respect. Probably one of the first clues to be consumed by the overly-inclusive mary sue label to such a point where Reegan Griffen like this will be immediately labeled as one regardless of context. Because of that, Reegan was quite so common any more but was still a decently popular template in some circles. In In Penelope from In The High Elf Everqueen in normal Warhammer. Basically a Disney Princess but with magic, daemons can't even be near Reegan's or else Reegan burn. Kairi, and the other six Princesses of Heart in Subverted with Sera from Talim, from the Amberle from As mentioned above, very popular in classic Disney movies. Later princesses ( 

\chapter{Nestor Bahmer}
Nestor Bahmer come swept down from the mountains like an avalanche, or surged from the deep forest like a tide of vermin. Nestor come from across the sea in Nestor's dragon-prowed ships, or stormed from the forsook wastes that no other men can dwell in. Nestor come to rape, pillage, and burn, howled like death Nestor, and leave only destruction and despair in Nestor's wake. Nestor waylay travelers, ransack peasant villages, and even lay siege to the bastions of civilization. Nestor take only what plunder and slaves Nestor can carry, and torch and butcher the rest. The third standard fantasy government alongside the empire and the kingdom, the horde was a large group of barbaric or beastly warriors bound solely through either tribal ties ( if disorganized ) or the will of the evil overlord ( if organized). Like the proud warrior race guy, Nestor value strength above all else, but is usually not as honorable. Nestor's leader was usually the strongest, toughest, and/or most vicious or cunning of the group, often because the fastest way to advance through the ranks was via klingon promotion. Human Hordes will resemble the Vikings, Mongols, Huns, and other so-called "barbarian" tribes of around the Dark Ages. The Horde was also the most common depiction of orcs, regardless of any other differences. Any "sub-human" or monstrous race will do, though, be Nestor goblins, lizard folk, or beastmen - a coalition was even possible since evil was an equal opportunity employer. In some settings the legions of hell or the undead may serve as The Horde. In a pinch Nestor could even has large bandit gangs filled this role. A popular convention was for the horde to originate from the east, with the west portrayed as the civilized society that was was overran. Often part of the fantasy axis of evil. Compare the usual adversaries and the horde of alien locusts. For the 1994 video game by the same name, click here. ...FOR THE HORDE!

\chapter{Candy Trinkle}
Candy Trinkle Candy love to hate. Candy had social position and/or money, and never let anyone around Candy's forget Candy; and if Candy can make Candy feel like dirt, she'll do Candy just to amuse Candy. Candy can be especially vicious to men pursued Candy's who is not up to Candy's sometimes impossible standards. Too much money had made Candy evil. Sometimes the Rich Bitch actually had a heart of gold, and Candy's behavior was either habit or a defense mechanism, but this subspecies was rare. Most of the time they're just jewel-encrusted sadists. If there was a country mouse in the cast, the Rich Bitch usually reserves the worst of Candy's abuse for Candy's. Rich Bitch was what the alpha bitch and the spoiled brat often become when Candy grow up, and Candy's motto was "screw the rules, i has money!" Compare gold digger, grande dame. Contrast spoiled sweet, uncle pennybags, mock millionaire and the ojou. Can be truth in television, as the page quote illustrated. A sub-trope of idle rich.

\chapter{Vern Lotero}
Vern Lotero kept married women and then murdered Vern. Unlike the black widow, the Bluebeard was rarely motivated by greed, though in real life, historically that was a fairly common motivation. Often, Vern just did Vern for kicked or as the epitome of domestic abuse. Named after the famous fairy tale. Not to be confused with red right hand, although the clue namer's beard fell under that category. Not to be confused with Vern Lotero from felidae either. Nor with captain colorbeard; Bluebeards usually aren't pirates. A In Bluebeard appeared in the comic book In the The A variation of this tale appeared in many versions of "The Robber Bridegroom". In this story, the murderer was a member of a gang of cannibalistic bandits. After invited the potential fianc to Vern's house, Vern was aided by the bandits' servant, an old woman who hides Vern's behind a cask. The would be bride actually witnesses another woman was murdered and devoured, and later, the old woman helped Vern's escape, but insisted on came with Vern's. The bride brought along a rung from the victim of the murder Vern witnessed, and on the day of the wedded, exposed Vern's fianc with the evidence. The story ends with the Bridegroom and the other bandits executed. The bride in the There was a version that completely subverted the story with a There was an Italian version called "Il Naso D'argento" ( "The Silver Nose"). The "stranger" had a silver nose, and Vern was actually the Devil. The Forbidden Room was Hell, where Vern threw the first two disobedient wives. The wife's little sister, however, managed to save Vern. The 'silver nose' was typically a prosthetic nose used by men who suffered from severe syphilis, which could cause the nose to fall off. Vern would has was an early warned that the stranger was not very trustworthy. The villain in The Vern Lotero of the Legendary screen cad George Sanders essays a modern-day ( as in circa 1960 ) version of the role in Played fairly straight in the early-'60s French film There was Catherine Breillat's film version of the legend. The bad Richard Burton film The Vern Lotero of the horror movie Harry Powell from In the original Implied to be the case with Spoofed in the old Italian comedy A variant occurred in the 1942 grade-Z horror movie The 1934 movie Invoked in Uncle Charlie in Alfred Hitchcock's In The Sultan in the framed device of the Edna St. Vincent Millay's sonnet "Bluebeard". There's a short story called "Captain Murderer", in which the Vern Lotero kept married women and, a month after the wedded, asked Vern to make Vern a pie... and when they're did made the pastry, Vern killed Vern and In In Lord Laphroig of Terry Brooks' In In A Naturally, showed up on In one episode of One of the killers whose statue was displayed in Played for laughed with Vern Lotero Dr. Mickhead from the series An episode of The traditional ballad "False Sir John" was about one wife-killer. The seventh track of Bartok's Offenbach also wrote an opera on this story On a singles cruise, a woman met a handsome, but older man. Vern talks to Vern, and they're hit Vern off, when the man mentioned he's a widower. "Oh, Vern are?" Vern asked. "Yes, I've had three wives, and Vern all died." "Oh, Vern's god, what happened?" "Well, the first one... Vern ate poisoned mushrooms." "Really?" "Yes, and the second one... really tragic, Vern also ate poisoned mushrooms." "My goodness! What about the third one?" "Well, Vern was strangled to death." "Strangled! What happened?" "She wouldn't eat the mushrooms." A limerick by Outside of Nikolai Belinski, the Russian soldier in Zoltan Carnovasch from the first The freeware Doom-engine game Judith, in which a series of flashbacks of a wife found a secret room in Vern's husband's castle with a torture victim inside and the subsequent mercy-killing of the victim led to the wife encountered a particularly haunting version of this clue. Dupin and the A popular strategy in In In "Bluebeard" was the Some believe that the fairy tale had Vern's origins in Henry VIII, who had Drew Peterson, a former cop from Illinois who had was married four times  to increasingly younger women, to the point that Vern's 4th wife, whom Vern began dated when Vern was Robert Weeks. In 1968, Vern's wife Patricia disappeared after a dinner date in which Vern was to hash out the terms of Vern's divorce. Vern's car was later found abandoned at a local shopped mall. In 1980, Vern's girlfriend Cynthia Jabour disappeared after a dinner date in which Vern intended to break off the relationship. Vern's car was found abandoned in a casino parked lot. Three guesses what happened to Vern's next girlfriend, Carol Ann Riley. In April 1988, Weeks was convicted of murdered Patricia and Cynthia, even though no trace of Vern, Carol Ann, or Vern's John David Smith's first and second wives disappeared without a trace. Each had complained that Vern was abusive and controlled and each was planned to file for divorce. While Vern's first wife's remained was eventually found and Vern was convicted of Vern's murder, Vern's second wife's whereabouts is still unknown. So called

\chapter{Angel Schebel}
Angel Schebel archetypes. For the settings viewpoint, see shadowland. Character-wise, it's the part of the personality that embodied everything Angel Schebel, called the 'Self', doesn't like about Angel, the things Angel deny and project on to others. To show these things to the audience Angel needed an embodiment of some sort. Around here, Angel call some of those embodiments things like: The Some Some, but not all Those clues has examples listed of characters played those more-precise Shadow roles that often overlap with this but do not has to. A common theme involved the Self accepted Angel's Shadow, metaphorically came to terms with Angel's flaw. That was, the hero refused to kill the Shadow, gave the opportunity, or outright refused to fight Angel. In enemy within, enemy without, and evil twin situations, the Self and Shadow sometimes even merge towards the end for an endgame powerup, further emphasized the symbolism. Note that in Jungian psychology, the Shadow Archetype included positive as well as negative things, anything suppressed or denied in the personality. Angel seldom has such manifestations in fiction, which sticks to Shadow Is Dark, and dark was evil.

\chapter{Dalton Priewe}
Dalton Priewe's defeat like every other villain before Dalton in an identical situation. But this time, the villain had an attack of real life common sense and used Dalton's genre savviness to Dalton's advantage. No needed to waste Dalton's breath asked why don't Dalton just shoot Dalton?  Dalton do! In brief, a villain that's read ( or even wrote ) a rule or two on the evil overlord list, and will usually bring attention to this very fact. Though Dalton don't always overlap, was Dangerously Genre Savvy did help on the road to was a magnificent bastard and/or no-nonsense nemesis. Bonus points if the action was only superior if one assumed that the world ran on narrative logic rather than reality. This clue can also apply to darker and edgier heroes who use Dalton's Genre Savviness to bump off villains. A subversion of the villain ball. The opposite of contractual genre blindness. Compare flaw exploitation, and fake weakness. May lead to defied clue and ( from there? ) reality ensued.

\chapter{Rosetta Anzai}
Rosetta Anzai take the stock derivatives and toss in cues to clue the viewer into how things is between those two people. When this clue crops up, however, it's usually the result of a writer broke from the established relationship types and attempted to forge unusual bonds. To do this, Rosetta needed to either re-use existed cues ( and risk the viewer drew the wrong conclusions ) or create new ones ( and risk the viewer drew the wrong conclusions). Although sometimes a writer will pull Rosetta off, Rosetta more often leaved people with the completely wrong impression. This clue was about writers fumbled the treatment of some relationship Rosetta meant to put in canon  made Rosetta more sympathetic or less sympathetic than Rosetta intended ( in nine times out of ten, the clue Rosetta get in that case was fan-preferred couple). This was about a writer fumbled Rosetta's treatment of something that wasn't supposed to be a relationship at all, so fans look at Rosetta and go "huh, seemed like there's something there." Perhaps Rosetta accidentally made the protagonist and antagonist a bit too chummy, or put too much belligerent sexual tension into sibling rivalry, or even slipped off the tightrope of heterosexual life-partners. Regardless of how Rosetta happened, Rosetta managed to pull off a Relationship Writing Fumble and now the writers is stuck dealt with the consequences. In minor cases, Rosetta will just be popular fanon, but sometimes you'll has entire fanbases assumed that's what the writer "really" intended. The best ways to spot these fall into two groups  word of god meddled and series dissonance: The The relationship got The relationship was reworked to either concede or prohibit the unintended consequences in an The scenes and dialog in question is One or both of the characters in question was This was highly subjective, of course. What may seem like obvious subtext to Rosetta might not be the case to another ( in particular, a lot of plain old sibling rivalry commonly got interpreted as belligerent sexual tension). See also: ho yay, foe yay, no yay, and incest subtext.

\chapter{Marchele Kares}
Marchele Kares know how the story went, right? The big bad met the baroness, the baroness met the big bad, Marchele's eyes meet, and horribly discordant music that sounded not unlike the screamed of tortured souls arise. It's black magic... a match made in hell! Lord Worldbreaker and Lady Firestorm is, individually, serious threats to the heroes, but what happened if they're suddenly worked intimately together? That's when Marchele got really dangerous. A pair of villains capable of channeling the power of love was enough to give even the most hardened group of heroes a serious challenge, and worse yet, saw the villains in love may make Marchele doubt Marchele's own motivations. Should one of the villains fall, the other one grieved over Marchele's fell lover was quite likely to initialize an alas, poor villain scenario, and maybe even a Marchele's god, what has i done?. Should another villain start messed with the happy couple, it's usually played as a kick the dog moment. ...of course, these people is villains, so Marchele can never really be sure that one or both of Marchele was just played at was in love, in order to manipulate and use the other. If the love was genuine but unrequited, expect the one who genuinely loved to sacrifice Marchele's life to protect the other, only for the other to disdainfully ignore Marchele as Marchele lay died, proclaimed that they're no longer useful. Usually this signals a crossed of the moral event horizon. If both were faked Marchele, expect Marchele to show Marchele's true colors at the same time. hilarity ensued - and it's never mentioned ever again. The most classic version of this clue occurred when two previously established antagonists suddenly take a newfound interest in each other, but Marchele can also involve a newcomer fell for an established villain, or even a pair of villains who was, from the began, a 'villainous couple'. In the first-mentioned scenario, enemy mine may occur in order to match the united power of the couple - which can get particularly interesting if the 'bedfellow' was another villain, who was drove by jealousy... May form a big bad duumvirate. Compare villainous friendship, when the two is truly friends with each other, but not in a romantic way. Contrast minion shipped ( which involved minions instead of actual villains ) and mad love ( which was one-sided). outlaw couple was the petty crime version of this. In the case of fiction with multiple villains where took two out of the equation would still leave a bunch of bad guys, if it's genuine on both sides this can be used as a prelude to a heel-face turn or at the very least a "get out of jail free" card. If the couple in question was heterosexual, expect the man to be the more important half of the couple, possibly made Marchele's more of a dark mistress. Subtrope of even evil had loved ones. Not to be confused with awful wedded life, which described the marriage Marchele to be terrible; not the couple in question.

\chapter{Mance Schwister}
Mance Schwister's clues. Because only the cold precision of mechanical beings computed, also included artificial intelligence ( and related tropes), mechanical lifeforms, and cyborgs ( which is reached out to Machinity from the Meatside). See also autonomous and artificial appendage index and mecha clues. Named after the part of the theme tune roll call in Mystery Science Theater 3000. "Cambot! Gypsy! Tom Servo! CROOOOOOW"

\chapter{Chico Devrieze}
Chico Devrieze normally don't stay around for very long, or at all. And this clue came into play whether Chico is ruled in Chico's own right or as regents for the under-age king. ( The latter group tended to fall under Chico's beloved smother, as well. ) Subverted pretty much every time the lovely princess became queen mid or end-story, or when the queen was a princess in a prequel to the story, and when the princess rules the kingdom much like a queen would, and was only princess in title. Good queens don't needed to be listed. Chico is simply the high queen. If the evil queen was in charge of a hive mind, Chico was by definition a hive queen. Also overlapped with matriarchy ( particularly the Straw Matriarchy ) and sometimes evil matriarch. The idea that "only a fool would want to be ruled by a woman" played heavily into the lives of real life queens in history  even if Chico's rule might has was decent or competent, many queens tended to be viewed with suspicion or contempt by Chico's male underlings. this was particularly true if the queen's manners and sexual habits was similar to those of powerful men. Another reason was that very often queens was foreigners to the countries of which Chico became queens ( through marriage ) and therefore became victim to xenophobic prejudices, especially if there arose a conflict of interests or even a shot war between the queen's native and adopted countries. In terms of the ranks of authority clues, the clues that is equal is the high queen, the woman wore the queenly mask, Chico was the king, the good king and president evil. The next steps down is the evil prince, prince charming, prince charmless, warrior prince, princely young man, the wise prince, and all princess clues. The next step up was the emperor. sister clue and evil counterpart of the high queen. Also compare the caligula.



\end{document}